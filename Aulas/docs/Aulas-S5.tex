\documentclass[11pt]{article}

    \usepackage[breakable]{tcolorbox}
    \usepackage{parskip} % Stop auto-indenting (to mimic markdown behaviour)
    
    \usepackage{iftex}
    \ifPDFTeX
    	\usepackage[T1]{fontenc}
    	\usepackage{mathpazo}
    \else
    	\usepackage{fontspec}
    \fi

    % Basic figure setup, for now with no caption control since it's done
    % automatically by Pandoc (which extracts ![](path) syntax from Markdown).
    \usepackage{graphicx}
    % Maintain compatibility with old templates. Remove in nbconvert 6.0
    \let\Oldincludegraphics\includegraphics
    % Ensure that by default, figures have no caption (until we provide a
    % proper Figure object with a Caption API and a way to capture that
    % in the conversion process - todo).
    \usepackage{caption}
    \DeclareCaptionFormat{nocaption}{}
    \captionsetup{format=nocaption,aboveskip=0pt,belowskip=0pt}

    \usepackage[Export]{adjustbox} % Used to constrain images to a maximum size
    \adjustboxset{max size={0.9\linewidth}{0.9\paperheight}}
    \usepackage{float}
    \floatplacement{figure}{H} % forces figures to be placed at the correct location
    \usepackage{xcolor} % Allow colors to be defined
    \usepackage{enumerate} % Needed for markdown enumerations to work
    \usepackage{geometry} % Used to adjust the document margins
    \usepackage{amsmath} % Equations
    \usepackage{amssymb} % Equations
    %\usepackage{mathtools}
    \usepackage{textcomp} % defines textquotesingle
    % Hack from http://tex.stackexchange.com/a/47451/13684:
    \AtBeginDocument{%
        \def\PYZsq{\textquotesingle}% Upright quotes in Pygmentized code
    }
    \usepackage{upquote} % Upright quotes for verbatim code
    \usepackage{eurosym} % defines \euro
    \usepackage[mathletters]{ucs} % Extended unicode (utf-8) support
    \usepackage{fancyvrb} % verbatim replacement that allows latex
    \usepackage{grffile} % extends the file name processing of package graphics 
                         % to support a larger range
    \makeatletter % fix for grffile with XeLaTeX
    \def\Gread@@xetex#1{%
      \IfFileExists{"\Gin@base".bb}%
      {\Gread@eps{\Gin@base.bb}}%
      {\Gread@@xetex@aux#1}%
    }
    \makeatother

    % The hyperref package gives us a pdf with properly built
    % internal navigation ('pdf bookmarks' for the table of contents,
    % internal cross-reference links, web links for URLs, etc.)
    \usepackage{hyperref}
    % The default LaTeX title has an obnoxious amount of whitespace. By default,
    % titling removes some of it. It also provides customization options.
    \usepackage{titling}
    \usepackage{longtable} % longtable support required by pandoc >1.10
    \usepackage{booktabs}  % table support for pandoc > 1.12.2
    \usepackage[inline]{enumitem} % IRkernel/repr support (it uses the enumerate* environment)
    \usepackage[normalem]{ulem} % ulem is needed to support strikethroughs (\sout)
                                % normalem makes italics be italics, not underlines
    \usepackage{mathrsfs}
    

    
    % Colors for the hyperref package
    \definecolor{urlcolor}{rgb}{0,.145,.698}
    \definecolor{linkcolor}{rgb}{.71,0.21,0.01}
    \definecolor{citecolor}{rgb}{.12,.54,.11}

    % ANSI colors
    \definecolor{ansi-black}{HTML}{3E424D}
    \definecolor{ansi-black-intense}{HTML}{282C36}
    \definecolor{ansi-red}{HTML}{E75C58}
    \definecolor{ansi-red-intense}{HTML}{B22B31}
    \definecolor{ansi-green}{HTML}{00A250}
    \definecolor{ansi-green-intense}{HTML}{007427}
    \definecolor{ansi-yellow}{HTML}{DDB62B}
    \definecolor{ansi-yellow-intense}{HTML}{B27D12}
    \definecolor{ansi-blue}{HTML}{208FFB}
    \definecolor{ansi-blue-intense}{HTML}{0065CA}
    \definecolor{ansi-magenta}{HTML}{D160C4}
    \definecolor{ansi-magenta-intense}{HTML}{A03196}
    \definecolor{ansi-cyan}{HTML}{60C6C8}
    \definecolor{ansi-cyan-intense}{HTML}{258F8F}
    \definecolor{ansi-white}{HTML}{C5C1B4}
    \definecolor{ansi-white-intense}{HTML}{A1A6B2}
    \definecolor{ansi-default-inverse-fg}{HTML}{FFFFFF}
    \definecolor{ansi-default-inverse-bg}{HTML}{000000}

    % commands and environments needed by pandoc snippets
    % extracted from the output of `pandoc -s`
    \providecommand{\tightlist}{%
      \setlength{\itemsep}{0pt}\setlength{\parskip}{0pt}}
    \DefineVerbatimEnvironment{Highlighting}{Verbatim}{commandchars=\\\{\}}
    % Add ',fontsize=\small' for more characters per line
    \newenvironment{Shaded}{}{}
    \newcommand{\KeywordTok}[1]{\textcolor[rgb]{0.00,0.44,0.13}{\textbf{{#1}}}}
    \newcommand{\DataTypeTok}[1]{\textcolor[rgb]{0.56,0.13,0.00}{{#1}}}
    \newcommand{\DecValTok}[1]{\textcolor[rgb]{0.25,0.63,0.44}{{#1}}}
    \newcommand{\BaseNTok}[1]{\textcolor[rgb]{0.25,0.63,0.44}{{#1}}}
    \newcommand{\FloatTok}[1]{\textcolor[rgb]{0.25,0.63,0.44}{{#1}}}
    \newcommand{\CharTok}[1]{\textcolor[rgb]{0.25,0.44,0.63}{{#1}}}
    \newcommand{\StringTok}[1]{\textcolor[rgb]{0.25,0.44,0.63}{{#1}}}
    \newcommand{\CommentTok}[1]{\textcolor[rgb]{0.38,0.63,0.69}{\textit{{#1}}}}
    \newcommand{\OtherTok}[1]{\textcolor[rgb]{0.00,0.44,0.13}{{#1}}}
    \newcommand{\AlertTok}[1]{\textcolor[rgb]{1.00,0.00,0.00}{\textbf{{#1}}}}
    \newcommand{\FunctionTok}[1]{\textcolor[rgb]{0.02,0.16,0.49}{{#1}}}
    \newcommand{\RegionMarkerTok}[1]{{#1}}
    \newcommand{\ErrorTok}[1]{\textcolor[rgb]{1.00,0.00,0.00}{\textbf{{#1}}}}
    \newcommand{\NormalTok}[1]{{#1}}
    
    % Additional commands for more recent versions of Pandoc
    \newcommand{\ConstantTok}[1]{\textcolor[rgb]{0.53,0.00,0.00}{{#1}}}
    \newcommand{\SpecialCharTok}[1]{\textcolor[rgb]{0.25,0.44,0.63}{{#1}}}
    \newcommand{\VerbatimStringTok}[1]{\textcolor[rgb]{0.25,0.44,0.63}{{#1}}}
    \newcommand{\SpecialStringTok}[1]{\textcolor[rgb]{0.73,0.40,0.53}{{#1}}}
    \newcommand{\ImportTok}[1]{{#1}}
    \newcommand{\DocumentationTok}[1]{\textcolor[rgb]{0.73,0.13,0.13}{\textit{{#1}}}}
    \newcommand{\AnnotationTok}[1]{\textcolor[rgb]{0.38,0.63,0.69}{\textbf{\textit{{#1}}}}}
    \newcommand{\CommentVarTok}[1]{\textcolor[rgb]{0.38,0.63,0.69}{\textbf{\textit{{#1}}}}}
    \newcommand{\VariableTok}[1]{\textcolor[rgb]{0.10,0.09,0.49}{{#1}}}
    \newcommand{\ControlFlowTok}[1]{\textcolor[rgb]{0.00,0.44,0.13}{\textbf{{#1}}}}
    \newcommand{\OperatorTok}[1]{\textcolor[rgb]{0.40,0.40,0.40}{{#1}}}
    \newcommand{\BuiltInTok}[1]{{#1}}
    \newcommand{\ExtensionTok}[1]{{#1}}
    \newcommand{\PreprocessorTok}[1]{\textcolor[rgb]{0.74,0.48,0.00}{{#1}}}
    \newcommand{\AttributeTok}[1]{\textcolor[rgb]{0.49,0.56,0.16}{{#1}}}
    \newcommand{\InformationTok}[1]{\textcolor[rgb]{0.38,0.63,0.69}{\textbf{\textit{{#1}}}}}
    \newcommand{\WarningTok}[1]{\textcolor[rgb]{0.38,0.63,0.69}{\textbf{\textit{{#1}}}}}
    
    
    % Define a nice break command that doesn't care if a line doesn't already
    % exist.
    \def\br{\hspace*{\fill} \\* }
    % Math Jax compatibility definitions
    \def\gt{>}
    \def\lt{<}
    \let\Oldtex\TeX
    \let\Oldlatex\LaTeX
    \renewcommand{\TeX}{\textrm{\Oldtex}}
    \renewcommand{\LaTeX}{\textrm{\Oldlatex}}
    % Document parameters
    % Document title
    \title{Aulas-S5}
    
    
    
    
    
% Pygments definitions
\makeatletter
\def\PY@reset{\let\PY@it=\relax \let\PY@bf=\relax%
    \let\PY@ul=\relax \let\PY@tc=\relax%
    \let\PY@bc=\relax \let\PY@ff=\relax}
\def\PY@tok#1{\csname PY@tok@#1\endcsname}
\def\PY@toks#1+{\ifx\relax#1\empty\else%
    \PY@tok{#1}\expandafter\PY@toks\fi}
\def\PY@do#1{\PY@bc{\PY@tc{\PY@ul{%
    \PY@it{\PY@bf{\PY@ff{#1}}}}}}}
\def\PY#1#2{\PY@reset\PY@toks#1+\relax+\PY@do{#2}}

\expandafter\def\csname PY@tok@w\endcsname{\def\PY@tc##1{\textcolor[rgb]{0.73,0.73,0.73}{##1}}}
\expandafter\def\csname PY@tok@c\endcsname{\let\PY@it=\textit\def\PY@tc##1{\textcolor[rgb]{0.25,0.50,0.50}{##1}}}
\expandafter\def\csname PY@tok@cp\endcsname{\def\PY@tc##1{\textcolor[rgb]{0.74,0.48,0.00}{##1}}}
\expandafter\def\csname PY@tok@k\endcsname{\let\PY@bf=\textbf\def\PY@tc##1{\textcolor[rgb]{0.00,0.50,0.00}{##1}}}
\expandafter\def\csname PY@tok@kp\endcsname{\def\PY@tc##1{\textcolor[rgb]{0.00,0.50,0.00}{##1}}}
\expandafter\def\csname PY@tok@kt\endcsname{\def\PY@tc##1{\textcolor[rgb]{0.69,0.00,0.25}{##1}}}
\expandafter\def\csname PY@tok@o\endcsname{\def\PY@tc##1{\textcolor[rgb]{0.40,0.40,0.40}{##1}}}
\expandafter\def\csname PY@tok@ow\endcsname{\let\PY@bf=\textbf\def\PY@tc##1{\textcolor[rgb]{0.67,0.13,1.00}{##1}}}
\expandafter\def\csname PY@tok@nb\endcsname{\def\PY@tc##1{\textcolor[rgb]{0.00,0.50,0.00}{##1}}}
\expandafter\def\csname PY@tok@nf\endcsname{\def\PY@tc##1{\textcolor[rgb]{0.00,0.00,1.00}{##1}}}
\expandafter\def\csname PY@tok@nc\endcsname{\let\PY@bf=\textbf\def\PY@tc##1{\textcolor[rgb]{0.00,0.00,1.00}{##1}}}
\expandafter\def\csname PY@tok@nn\endcsname{\let\PY@bf=\textbf\def\PY@tc##1{\textcolor[rgb]{0.00,0.00,1.00}{##1}}}
\expandafter\def\csname PY@tok@ne\endcsname{\let\PY@bf=\textbf\def\PY@tc##1{\textcolor[rgb]{0.82,0.25,0.23}{##1}}}
\expandafter\def\csname PY@tok@nv\endcsname{\def\PY@tc##1{\textcolor[rgb]{0.10,0.09,0.49}{##1}}}
\expandafter\def\csname PY@tok@no\endcsname{\def\PY@tc##1{\textcolor[rgb]{0.53,0.00,0.00}{##1}}}
\expandafter\def\csname PY@tok@nl\endcsname{\def\PY@tc##1{\textcolor[rgb]{0.63,0.63,0.00}{##1}}}
\expandafter\def\csname PY@tok@ni\endcsname{\let\PY@bf=\textbf\def\PY@tc##1{\textcolor[rgb]{0.60,0.60,0.60}{##1}}}
\expandafter\def\csname PY@tok@na\endcsname{\def\PY@tc##1{\textcolor[rgb]{0.49,0.56,0.16}{##1}}}
\expandafter\def\csname PY@tok@nt\endcsname{\let\PY@bf=\textbf\def\PY@tc##1{\textcolor[rgb]{0.00,0.50,0.00}{##1}}}
\expandafter\def\csname PY@tok@nd\endcsname{\def\PY@tc##1{\textcolor[rgb]{0.67,0.13,1.00}{##1}}}
\expandafter\def\csname PY@tok@s\endcsname{\def\PY@tc##1{\textcolor[rgb]{0.73,0.13,0.13}{##1}}}
\expandafter\def\csname PY@tok@sd\endcsname{\let\PY@it=\textit\def\PY@tc##1{\textcolor[rgb]{0.73,0.13,0.13}{##1}}}
\expandafter\def\csname PY@tok@si\endcsname{\let\PY@bf=\textbf\def\PY@tc##1{\textcolor[rgb]{0.73,0.40,0.53}{##1}}}
\expandafter\def\csname PY@tok@se\endcsname{\let\PY@bf=\textbf\def\PY@tc##1{\textcolor[rgb]{0.73,0.40,0.13}{##1}}}
\expandafter\def\csname PY@tok@sr\endcsname{\def\PY@tc##1{\textcolor[rgb]{0.73,0.40,0.53}{##1}}}
\expandafter\def\csname PY@tok@ss\endcsname{\def\PY@tc##1{\textcolor[rgb]{0.10,0.09,0.49}{##1}}}
\expandafter\def\csname PY@tok@sx\endcsname{\def\PY@tc##1{\textcolor[rgb]{0.00,0.50,0.00}{##1}}}
\expandafter\def\csname PY@tok@m\endcsname{\def\PY@tc##1{\textcolor[rgb]{0.40,0.40,0.40}{##1}}}
\expandafter\def\csname PY@tok@gh\endcsname{\let\PY@bf=\textbf\def\PY@tc##1{\textcolor[rgb]{0.00,0.00,0.50}{##1}}}
\expandafter\def\csname PY@tok@gu\endcsname{\let\PY@bf=\textbf\def\PY@tc##1{\textcolor[rgb]{0.50,0.00,0.50}{##1}}}
\expandafter\def\csname PY@tok@gd\endcsname{\def\PY@tc##1{\textcolor[rgb]{0.63,0.00,0.00}{##1}}}
\expandafter\def\csname PY@tok@gi\endcsname{\def\PY@tc##1{\textcolor[rgb]{0.00,0.63,0.00}{##1}}}
\expandafter\def\csname PY@tok@gr\endcsname{\def\PY@tc##1{\textcolor[rgb]{1.00,0.00,0.00}{##1}}}
\expandafter\def\csname PY@tok@ge\endcsname{\let\PY@it=\textit}
\expandafter\def\csname PY@tok@gs\endcsname{\let\PY@bf=\textbf}
\expandafter\def\csname PY@tok@gp\endcsname{\let\PY@bf=\textbf\def\PY@tc##1{\textcolor[rgb]{0.00,0.00,0.50}{##1}}}
\expandafter\def\csname PY@tok@go\endcsname{\def\PY@tc##1{\textcolor[rgb]{0.53,0.53,0.53}{##1}}}
\expandafter\def\csname PY@tok@gt\endcsname{\def\PY@tc##1{\textcolor[rgb]{0.00,0.27,0.87}{##1}}}
\expandafter\def\csname PY@tok@err\endcsname{\def\PY@bc##1{\setlength{\fboxsep}{0pt}\fcolorbox[rgb]{1.00,0.00,0.00}{1,1,1}{\strut ##1}}}
\expandafter\def\csname PY@tok@kc\endcsname{\let\PY@bf=\textbf\def\PY@tc##1{\textcolor[rgb]{0.00,0.50,0.00}{##1}}}
\expandafter\def\csname PY@tok@kd\endcsname{\let\PY@bf=\textbf\def\PY@tc##1{\textcolor[rgb]{0.00,0.50,0.00}{##1}}}
\expandafter\def\csname PY@tok@kn\endcsname{\let\PY@bf=\textbf\def\PY@tc##1{\textcolor[rgb]{0.00,0.50,0.00}{##1}}}
\expandafter\def\csname PY@tok@kr\endcsname{\let\PY@bf=\textbf\def\PY@tc##1{\textcolor[rgb]{0.00,0.50,0.00}{##1}}}
\expandafter\def\csname PY@tok@bp\endcsname{\def\PY@tc##1{\textcolor[rgb]{0.00,0.50,0.00}{##1}}}
\expandafter\def\csname PY@tok@fm\endcsname{\def\PY@tc##1{\textcolor[rgb]{0.00,0.00,1.00}{##1}}}
\expandafter\def\csname PY@tok@vc\endcsname{\def\PY@tc##1{\textcolor[rgb]{0.10,0.09,0.49}{##1}}}
\expandafter\def\csname PY@tok@vg\endcsname{\def\PY@tc##1{\textcolor[rgb]{0.10,0.09,0.49}{##1}}}
\expandafter\def\csname PY@tok@vi\endcsname{\def\PY@tc##1{\textcolor[rgb]{0.10,0.09,0.49}{##1}}}
\expandafter\def\csname PY@tok@vm\endcsname{\def\PY@tc##1{\textcolor[rgb]{0.10,0.09,0.49}{##1}}}
\expandafter\def\csname PY@tok@sa\endcsname{\def\PY@tc##1{\textcolor[rgb]{0.73,0.13,0.13}{##1}}}
\expandafter\def\csname PY@tok@sb\endcsname{\def\PY@tc##1{\textcolor[rgb]{0.73,0.13,0.13}{##1}}}
\expandafter\def\csname PY@tok@sc\endcsname{\def\PY@tc##1{\textcolor[rgb]{0.73,0.13,0.13}{##1}}}
\expandafter\def\csname PY@tok@dl\endcsname{\def\PY@tc##1{\textcolor[rgb]{0.73,0.13,0.13}{##1}}}
\expandafter\def\csname PY@tok@s2\endcsname{\def\PY@tc##1{\textcolor[rgb]{0.73,0.13,0.13}{##1}}}
\expandafter\def\csname PY@tok@sh\endcsname{\def\PY@tc##1{\textcolor[rgb]{0.73,0.13,0.13}{##1}}}
\expandafter\def\csname PY@tok@s1\endcsname{\def\PY@tc##1{\textcolor[rgb]{0.73,0.13,0.13}{##1}}}
\expandafter\def\csname PY@tok@mb\endcsname{\def\PY@tc##1{\textcolor[rgb]{0.40,0.40,0.40}{##1}}}
\expandafter\def\csname PY@tok@mf\endcsname{\def\PY@tc##1{\textcolor[rgb]{0.40,0.40,0.40}{##1}}}
\expandafter\def\csname PY@tok@mh\endcsname{\def\PY@tc##1{\textcolor[rgb]{0.40,0.40,0.40}{##1}}}
\expandafter\def\csname PY@tok@mi\endcsname{\def\PY@tc##1{\textcolor[rgb]{0.40,0.40,0.40}{##1}}}
\expandafter\def\csname PY@tok@il\endcsname{\def\PY@tc##1{\textcolor[rgb]{0.40,0.40,0.40}{##1}}}
\expandafter\def\csname PY@tok@mo\endcsname{\def\PY@tc##1{\textcolor[rgb]{0.40,0.40,0.40}{##1}}}
\expandafter\def\csname PY@tok@ch\endcsname{\let\PY@it=\textit\def\PY@tc##1{\textcolor[rgb]{0.25,0.50,0.50}{##1}}}
\expandafter\def\csname PY@tok@cm\endcsname{\let\PY@it=\textit\def\PY@tc##1{\textcolor[rgb]{0.25,0.50,0.50}{##1}}}
\expandafter\def\csname PY@tok@cpf\endcsname{\let\PY@it=\textit\def\PY@tc##1{\textcolor[rgb]{0.25,0.50,0.50}{##1}}}
\expandafter\def\csname PY@tok@c1\endcsname{\let\PY@it=\textit\def\PY@tc##1{\textcolor[rgb]{0.25,0.50,0.50}{##1}}}
\expandafter\def\csname PY@tok@cs\endcsname{\let\PY@it=\textit\def\PY@tc##1{\textcolor[rgb]{0.25,0.50,0.50}{##1}}}

\def\PYZbs{\char`\\}
\def\PYZus{\char`\_}
\def\PYZob{\char`\{}
\def\PYZcb{\char`\}}
\def\PYZca{\char`\^}
\def\PYZam{\char`\&}
\def\PYZlt{\char`\<}
\def\PYZgt{\char`\>}
\def\PYZsh{\char`\#}
\def\PYZpc{\char`\%}
\def\PYZdl{\char`\$}
\def\PYZhy{\char`\-}
\def\PYZsq{\char`\'}
\def\PYZdq{\char`\"}
\def\PYZti{\char`\~}
% for compatibility with earlier versions
\def\PYZat{@}
\def\PYZlb{[}
\def\PYZrb{]}
\makeatother


    % For linebreaks inside Verbatim environment from package fancyvrb. 
    \makeatletter
        \newbox\Wrappedcontinuationbox 
        \newbox\Wrappedvisiblespacebox 
        \newcommand*\Wrappedvisiblespace {\textcolor{red}{\textvisiblespace}} 
        \newcommand*\Wrappedcontinuationsymbol {\textcolor{red}{\llap{\tiny$\m@th\hookrightarrow$}}} 
        \newcommand*\Wrappedcontinuationindent {3ex } 
        \newcommand*\Wrappedafterbreak {\kern\Wrappedcontinuationindent\copy\Wrappedcontinuationbox} 
        % Take advantage of the already applied Pygments mark-up to insert 
        % potential linebreaks for TeX processing. 
        %        {, <, #, %, $, ' and ": go to next line. 
        %        _, }, ^, &, >, - and ~: stay at end of broken line. 
        % Use of \textquotesingle for straight quote. 
        \newcommand*\Wrappedbreaksatspecials {% 
            \def\PYGZus{\discretionary{\char`\_}{\Wrappedafterbreak}{\char`\_}}% 
            \def\PYGZob{\discretionary{}{\Wrappedafterbreak\char`\{}{\char`\{}}% 
            \def\PYGZcb{\discretionary{\char`\}}{\Wrappedafterbreak}{\char`\}}}% 
            \def\PYGZca{\discretionary{\char`\^}{\Wrappedafterbreak}{\char`\^}}% 
            \def\PYGZam{\discretionary{\char`\&}{\Wrappedafterbreak}{\char`\&}}% 
            \def\PYGZlt{\discretionary{}{\Wrappedafterbreak\char`\<}{\char`\<}}% 
            \def\PYGZgt{\discretionary{\char`\>}{\Wrappedafterbreak}{\char`\>}}% 
            \def\PYGZsh{\discretionary{}{\Wrappedafterbreak\char`\#}{\char`\#}}% 
            \def\PYGZpc{\discretionary{}{\Wrappedafterbreak\char`\%}{\char`\%}}% 
            \def\PYGZdl{\discretionary{}{\Wrappedafterbreak\char`\$}{\char`\$}}% 
            \def\PYGZhy{\discretionary{\char`\-}{\Wrappedafterbreak}{\char`\-}}% 
            \def\PYGZsq{\discretionary{}{\Wrappedafterbreak\textquotesingle}{\textquotesingle}}% 
            \def\PYGZdq{\discretionary{}{\Wrappedafterbreak\char`\"}{\char`\"}}% 
            \def\PYGZti{\discretionary{\char`\~}{\Wrappedafterbreak}{\char`\~}}% 
        } 
        % Some characters . , ; ? ! / are not pygmentized. 
        % This macro makes them "active" and they will insert potential linebreaks 
        \newcommand*\Wrappedbreaksatpunct {% 
            \lccode`\~`\.\lowercase{\def~}{\discretionary{\hbox{\char`\.}}{\Wrappedafterbreak}{\hbox{\char`\.}}}% 
            \lccode`\~`\,\lowercase{\def~}{\discretionary{\hbox{\char`\,}}{\Wrappedafterbreak}{\hbox{\char`\,}}}% 
            \lccode`\~`\;\lowercase{\def~}{\discretionary{\hbox{\char`\;}}{\Wrappedafterbreak}{\hbox{\char`\;}}}% 
            \lccode`\~`\:\lowercase{\def~}{\discretionary{\hbox{\char`\:}}{\Wrappedafterbreak}{\hbox{\char`\:}}}% 
            \lccode`\~`\?\lowercase{\def~}{\discretionary{\hbox{\char`\?}}{\Wrappedafterbreak}{\hbox{\char`\?}}}% 
            \lccode`\~`\!\lowercase{\def~}{\discretionary{\hbox{\char`\!}}{\Wrappedafterbreak}{\hbox{\char`\!}}}% 
            \lccode`\~`\/\lowercase{\def~}{\discretionary{\hbox{\char`\/}}{\Wrappedafterbreak}{\hbox{\char`\/}}}% 
            \catcode`\.\active
            \catcode`\,\active 
            \catcode`\;\active
            \catcode`\:\active
            \catcode`\?\active
            \catcode`\!\active
            \catcode`\/\active 
            \lccode`\~`\~ 	
        }
    \makeatother

    \let\OriginalVerbatim=\Verbatim
    \makeatletter
    \renewcommand{\Verbatim}[1][1]{%
        %\parskip\z@skip
        \sbox\Wrappedcontinuationbox {\Wrappedcontinuationsymbol}%
        \sbox\Wrappedvisiblespacebox {\FV@SetupFont\Wrappedvisiblespace}%
        \def\FancyVerbFormatLine ##1{\hsize\linewidth
            \vtop{\raggedright\hyphenpenalty\z@\exhyphenpenalty\z@
                \doublehyphendemerits\z@\finalhyphendemerits\z@
                \strut ##1\strut}%
        }%
        % If the linebreak is at a space, the latter will be displayed as visible
        % space at end of first line, and a continuation symbol starts next line.
        % Stretch/shrink are however usually zero for typewriter font.
        \def\FV@Space {%
            \nobreak\hskip\z@ plus\fontdimen3\font minus\fontdimen4\font
            \discretionary{\copy\Wrappedvisiblespacebox}{\Wrappedafterbreak}
            {\kern\fontdimen2\font}%
        }%
        
        % Allow breaks at special characters using \PYG... macros.
        \Wrappedbreaksatspecials
        % Breaks at punctuation characters . , ; ? ! and / need catcode=\active 	
        \OriginalVerbatim[#1,codes*=\Wrappedbreaksatpunct]%
    }
    \makeatother

    % Exact colors from NB
    \definecolor{incolor}{HTML}{303F9F}
    \definecolor{outcolor}{HTML}{D84315}
    \definecolor{cellborder}{HTML}{CFCFCF}
    \definecolor{cellbackground}{HTML}{F7F7F7}
    
    % prompt
    \makeatletter
    \newcommand{\boxspacing}{\kern\kvtcb@left@rule\kern\kvtcb@boxsep}
    \makeatother
    \newcommand{\prompt}[4]{
        \ttfamily\llap{{\color{#2}[#3]:\hspace{3pt}#4}}\vspace{-\baselineskip}
    }
    

    
    % Prevent overflowing lines due to hard-to-break entities
    \sloppy 
    % Setup hyperref package
    \hypersetup{
      breaklinks=true,  % so long urls are correctly broken across lines
      colorlinks=true,
      urlcolor=urlcolor,
      linkcolor=linkcolor,
      citecolor=citecolor,
      }
    % Slightly bigger margins than the latex defaults
    
    \geometry{verbose,tmargin=1in,bmargin=1in,lmargin=1in,rmargin=1in}
    
    

\begin{document}
    
    \maketitle
    
    

    
    \(\newcommand{\bra}[1]{\left\langle #1 \right|}\)
\(\newcommand{\ket}[1]{\left| #1 \right\rangle}\)
\(\newcommand{\braket}[3]{\left\langle #1 \middle| #2 \middle| #3 \right\rangle}\)

\hypertarget{estrutura-matemuxe1tica-da-mecuxe2nica-quuxe2ntica}{%
\section{5. Estrutura matemática da Mecânica
Quântica}\label{estrutura-matemuxe1tica-da-mecuxe2nica-quuxe2ntica}}

Até este ponto, discutimos, em linhas gerais, como expressar e resolver
problemas físicos na mecânica quântica, em termos da Equação de
Schrödinger (EqS). Discutimos, de uma maneira ampla, as estratégias para
resolver a EqS no caso geral e, em particular, discutimos a resolução da
equação independente do tempo, resolvendo alguns exemplos emblemáticos
de potencias unidimencionais simples. Visto sob essa perspectiva,
pode-se ter a impressão que mecânica quântica se resume à solução da
EqS, usando métodos matemáticos mais ou menos familiares (solução de
equações diferenciais parciais). Embora essa seja uma estratégia válida
e efetiva em alguns casos, ela é bastante limitada e seria um grande
equívoco pensar que as estratégias da mecânica quântica se limitam
simplesmente a soluções da Eq. de Schrödinger.

O roteiro seguido até aqui teve uma motivação didática e, deliberamente,
procurou enfatizar os aspectos físicos do problema. Apresentando apenas
a matemática necessária para formular e resolver o problema. Por essa
razão, não temos sido muito rigorosos com o formalismo. Trocando rigor
matemático por intuição física, sempre que possível, para não obscurecer
desnecessariamente a ``Física'' do problema. Essa estratégia é bastante
razoável para uma introdução ao assunto. Apesar disso, o domínio do
formalismo matemático também é importante e necessário para ser bem
sucedido na resolução de problemas gerais da MQ, ou mesmo para entender
muitos temas de pesquisa contemporânea. A situação ideal é aquela onde
consegue-se combinar ambas habilidades, que é um dos objetivos
secundários deste curso.

Neste capitulo, portanto, seguiremos uma estratégia diferente e
complementar àquela seguida até agora. O foco agora será ampliar a
linguagem e abstração do problema, apresentadno de modo mais formal a
estrutura matemática da mecânica quântica moderna. A prioridade ainda
permanecerá com a Física e não a Matemática. Portanto, não se almeja
mero rigor matemático, mas, sim, introduzir novos conceitos e
representações que serão muito úteis para expandir os horizontes dentro
da teoria e, como iremos explorar nos próximos capítulos, serão
fundamentais para entender a linguagem contemporânea dessa importante
disciplina científica.

\hypertarget{espauxe7o-de-estados}{%
\subsection{5.1 Espaço de estados}\label{espauxe7o-de-estados}}

Resumindo o que vimos até aqui, podemos, ainda de uma maneira informal,
dizer que as soluções estacionárias \( \psi_n(x) \) da EqS são funções de
ondas que representam os possíveis estados do sistema, com energia
\(E_n\). Outra forma de dizer isso, motivada pela forma da equação
\( H\psi_n(x)=E_n\psi_n(x) \), é dizer que \( \{ \psi_n(x) \} \) é o conjunto
de autofunções do operador (H), representando os autoestados do sistema
com autovalores \(E_n\). Vimos nos exemplos discutidos, como no caso da
caixa infinita, que \( \psi_n(x) \) possui uma série de propriedades
interessantes e úteis. Entre elas:

Dentro do que vimos até aqui, podemos, ainda de uma maneira informal,
dizer que as soluções estacionárias \(\psi_n(x)\) da EqS são funções de
ondas que representam os possíveis estados do sistema, com energia
\(E_n\). Outra forma de dizer isso, observando a forma da equação
\(H\psi_n(x)=E_n\psi_n(x)\), é dizer que \(\{\psi_n(x)\}\) é o conjunto
de autofunções do operador \(H\), representando os autoestados do
sistema com autovalores \(E_n\). Vimos nos exemplos discutidos, como no
caso da caixa infinita, que \(\psi_n(x)\) possui uma série de
propriedades interessantes e úteis. Entre elas:
\textgreater{}\(\int \psi^*_n(x)\psi_m(x)dx=\delta_{nm}\) \textgreater{}
\textgreater{}\(\Psi(x)=\sum_n c_n \psi_n(x)\)\\
\textgreater{} \textgreater{}\(c_n = \int \psi^*_n(x) \Psi(x)dx\) ; onde
\(\sum_n |c_n|^2 = 1\) \textgreater{}
\textgreater{}\(<A_{_{\Psi}}> = \int \Psi^*(x) A \Psi(x) dx\)

    De fato, pode-se extender e generalizar essas ideias para expressar
esses objetos em termos mais abstratos e gerais, através do conceito de
espaço vetorial linear. Como os estados \(\psi_n(x)\) e os operadores
(que nesse contexto serão transformações lineares) nesses estados devem
satisfazer um certas propriedades para representar um sistema físico,
esses espaços vetoriais devem ter conjunto de estruturas e propriedades
especiais que veremos logo mais. Por simplicidade, iremos nos referir a
esses espaços como \emph{espaços de Hilbert}.

    Para deixar esse ponto mais claro, vamos relembrar/introduzir algumas
definições e conceitos, para formalizar e definir melhor essa ideia.

    \hypertarget{espauxe7o-vetorial-linear}{%
\subsection{5.2 Espaço vetorial
linear}\label{espauxe7o-vetorial-linear}}

Partido da definição mais geral e abstrata:

\textbf{Definição 1} \textgreater{} Grupo comutativo sob adição,
\(\mathcal{V}\), com multiplicação por escalar definida sobre um campo
complexo \(\mathcal{F}\), satisfazendo propriedades associativa e
distributiva. Os elementos do espaço \(\mathcal{V}\) são chamados de
\emph{vetores} e os elementos do campo \(\mathcal{F}\) são
\emph{escalares}.

As propriedades associativa e distributiva da multiplicação por escalar
implica:

Se \(\mathcal{V}=\{\vec{u},\vec{v},\vec{w},...\}\) e
\(\mathcal{F}=\{\lambda,\mu,\kappa,...\}\), temos que:
\(\lambda(\mu\vec{v})=(\lambda\mu)\vec{v}\),\\
\(\lambda(\vec{v}+\vec{u})=\lambda\vec{v}+\lambda\vec{u}\) e
\((\lambda+\mu)\vec{u}=\lambda\vec{u}+\mu\vec{u}\).

Vale lembrar algumas outras definições (\emph{Grupo} e \emph{Campo}), da
Algebra:

\begin{quote}
\textbf{Grupo:} Conjunto de elementos, que inclui inversos e identidade,
com uma operação (\(*\)) fechada que satisfaz associatividade. Grupos
não precisam ser comutativos, mas quando apresentam essa propriedade são
chamados de grupos comutativos ou Abelianos. \textgreater{} 1.
\emph{Fechado}: \(\forall\, x,y \in G \rightarrow x*y \in G\)
\textgreater{} 2. \emph{Associativo}:
\(\forall\, x,y,z \in G \rightarrow (x*y)*z=x*(y*z)\) \textgreater{} 3.
\emph{Identidade}:
\(\exists\, e\in G \rightarrow e*x=x*e=x; \,\, \forall\, x \in G\)
\textgreater{} 4. \emph{Inverso}:
\(\forall\, x \in G, \exists\, x^{-1} \rightarrow (x^{-1})*x=x*(x^{-1})=e\)

\textbf{Campo:} De maneira simples, são conjuntos de elementos onde são
definidas as quatro operações aritméticas
(\(+\),\(-\),\(\times\),\(\div\)) de forma comutativa. Como as operações
(\(-\), \(\div\)) são, na verdade, operações inversas de
(\(+\),\(\times\)), são definidos em termos dessas duas operações.
\textgreater{} Formalmente, campos são conjuntos de elementos com
operações de adição e multiplicação (\(+\),\(\times\)) definida; sendo
comutativo para (\(+\)) e comutativo para (\(\times\)) omitindo o
elemento nulo (zero). Satisfaz ainda a propriedade distributiva
\(a\times(b+c)=a\times b + a\times c\).

Campos são, portanto, dois grupos comutativos com duas operações
(\(+\),\(\times\)). Exemplos importantes são os campos dos números
reais, complexos e racionais.
\end{quote}

Alternativamente, uma definição um pouco mais familiar de \textbf{espaço
vetorial} é:

\textbf{Definição 2:} \textgreater{} Conjunto
\(\mathcal{V}\ne\emptyset\) (não vazio) de elementos, chamados vetores,
que é fechado sob adição e multiplicação por um escalar de um campo
complexo \(\mathcal{F}\).

Ou seja, se \(\mathcal{V}=\{\vec{u},\vec{v},\vec{w},...\}\) e
\(\mathcal{F}=\{\lambda,\mu,\kappa,...\}\), temos que:
\(\forall\, \vec{u},\vec{w}\in \mathcal{V}\) e
\(\forall\, \lambda,\mu \in \mathcal{F} \rightarrow \lambda\vec{u}+\mu\vec{w} \in \mathcal{V}\)

Se o campo \(\mathcal{F}\) é complexo (real) o espaço é dito ser um
espaço vetorial linear complexo (real).

\hypertarget{dimensuxe3o-do-espauxe7o}{%
\subsubsection{Dimensão do espaço}\label{dimensuxe3o-do-espauxe7o}}

Um conjunto de vetores \(\{\phi_n \}\) é dito linearmente independente
(LI) se não há nenhuma combinação linear não-trivial que leve ao vetor
nulo, isto é:
\(\sum_n c_n \phi_n = 0 \rightarrow c_n = 0\, \forall\, n\). A dimensão
\(d\) do espaço vetorial é dada pelo número máximo de vetores LI desse
espaço. Qualquer vetor do espaço pode ser escrito como uma combinação
linear dos vetores da base desse espaço, formado por vetores LI do
espaço. Como veremos adiante, os espaçcos de Hilbert da MQ podem ser
infinitos.

\hypertarget{espauxe7os-de-hilbert-espauxe7os-vetoriais-da-mq}{%
\subsection{5.3 Espaços de Hilbert: espaços vetoriais da
MQ}\label{espauxe7os-de-hilbert-espauxe7os-vetoriais-da-mq}}

Na mecânica quântica são usados espaços vetoriais com algumas
propriedades e estruturas adicionais, para garantir certas propriedades
físicas desejáveis da teoria. É comum, principalmente entre os físicos,
chamar esses estados de estados de Hilbert. Os espaços de Hilbert podem
ser finitos (com dimensão (d)) ou infinitos, por exemplo, quando os
vetores são funções contínuas.

Embora essa terminologia não seja muito precisa, dado que os espaços
vetoriais usados na MQ são apenas um tipo particular de espaço de
Hilbert (neste contexto: os espaços cujos vetores são funções
\emph{quadrado-integráveis}, também chamados de espaços de Lebesgue do
tipo (L\_2)), nós usaremos essa ``convenção'', para simplificar a
linguagem.

\hypertarget{produto-interno}{%
\subsubsection{Produto interno}\label{produto-interno}}

Uma das estruturas adicionais dos espaços de Hilbert é o produto interno
que leva dois vetores do espaço num número complexo, segundo a
definição:

\[\forall\, \phi, \psi \in \mathcal{H} \rightarrow (\phi,\psi) = \int \phi^*(x)\psi(x)\,dx\]

No caso de um espaço discreto de dimensão (d), o produto interno é
definido como

\[(w,v)=\sum_{i=1}^{d} w_i^* v_i\]

Note que o produto interno resulta num escalar complexo, que não é um
elemento do espaço de Hilbert. O produto interno tem as seguintes
propriedades: \textgreater{} \((\phi,\psi) = \lambda \in \mathbb{C}\)
(número complexo) \textgreater{} \textgreater{}
\((\phi,\psi) = (\psi,\phi)^*\) \textgreater{} \textgreater{}
\((\phi, c_1 \psi_1 + c_2 \psi_2 ) = c_1(\phi, \psi_1) + c_2(\phi,\psi_2 )\)
\textgreater{} \textgreater{}
\((c_1 \psi_1 + c_2 \psi_2, \phi ) = c_1^* (\psi_1, \phi) + c_2^*(\psi_2, \phi)\)
\textgreater{} \textgreater{} \((\phi,\phi) \ge 0\), sendo nulo apenas
quando \(\phi=0\)

\hypertarget{comprimentos-e-uxe2ngulos}{%
\subsubsection{Comprimentos e ângulos}\label{comprimentos-e-uxe2ngulos}}

O conceito de produto interno nos permite generalizar os conceitos de
comprimento (norma) e medidas de ângulos entre vetores em espaços de
dimensões e elementos arbitrários. Embora os vetores agora não sejam
mais ``setas'' no espaço tridimensional Euclidiano, pode-se explorar a
analogia com o conceito de produto escalar (o produto interno) daquele
espaço, para definir a norma do vetor, através do produto interno de um
vetor por ele mesmo:

\[(\phi,\phi) = \int \phi^*(x)\phi(x)\,dx = |\phi|^2\]

\[(v,v) = \sum_{i=1}^{d} v^*_i v_i = |v|^2\]

\[||\phi|| = \sqrt{|\phi|^2}\]

\[||v|| = \sqrt{|v|^2}\]

Observe que a norma é sempre um número real, tal que \(||\phi|| \ge 0\)
e \(||v|| \ge 0\), conforme nos assegura a desigualdade de Schwartz:

\[ |(\psi,\phi)|^2 \le (\psi,\psi)(\phi,\phi).\]

Também é satisfeito o teorema de desigualdade triangular:

\[ ||(\psi + \phi)|| \le ||\psi|| + ||\phi|| .\]

Para ambos os casos, a desigualdade só é válida quando um dos vetores é
múltiplo do outro.

Dois veltores são tido ortogonais quando seu produto interno é nulo. Da
mesma forma, um conjunto de vetores \(\{\phi_n\}\) é dito
\emph{ortonormal} quando o produto interno entre pares de seus elementos
obedece a relação \((\phi_n,\phi_m)=\delta_{nm}\).

    \hypertarget{expansuxe3o-de-vetores}{%
\subsubsection{Expansão de vetores}\label{expansuxe3o-de-vetores}}

No caso em que \(\mathcal{H}\) é finito, com dimensão \(d\), dado um
vetor arbitrário \(\psi\) e uma base \(\{ \phi_n \}\) de vetores
linearmente independentes, podemos expressar o vetor
\(\psi = \sum_n c_n \phi_n\), onde \(c_n=(\phi_n,\psi)\) e
\((\phi_n,\phi_m)=\delta_{nm}\). Podemos pensar nos coeficientes \(c_n\)
como sendo as componentes do vetor no espaço de Hilbert, análogos às
componentes de um vetor no espaço Euclidiano. Porém, é importante
lembrar que essas componentes são expressas por números complexos. As
componente do vetor de estado têm toda a informação relativa ao estado,
determinando completamente o vetor (estado) do sistema.

Também de modo análogo, podemos expressar as soma de dois vetore em
termos dessas componentes

\[\Psi_a + \Psi_b = \sum_n (a_i + b_n) \psi_n.\]

\[\lambda \Psi_a= \sum_n \lambda a_i \psi_n.\]

    \begin{quote}
\textbf{\emph{Pare, Pense \& Contemple!}}

Antes de prosseguir, pare e reflita por um momento no significado e
amplitude esses resultados. Lembre-se que o espaço \(\mathcal{H}\) pode
ter dimensõe infinitas, tanto no número de elemento (vetores), como nas
dimensões (número de componentes) desses vetores. Esses resultados, nada
óbvios, são extremamente poderosos e úteis, justificando plenamente o
tempo investido em generalizar e abstrair a descrição dos nossos
problemas usando esse formalismo.
\end{quote}

    \hypertarget{notauxe7uxe3o-de-dirac}{%
\subsection{5.4 Notação de Dirac}\label{notauxe7uxe3o-de-dirac}}

Introduzimos agora a notação de Dirac, bastante popular na mecânica
quântica, onde o vetor de estado é chamado de ``\emph{ket}'' e
representado pelo símbolo \(|\psi\rangle\). O vetor correspondente do
\emph{espaço dual} é chamado de ``\emph{bra}'' é representado por
\(\langle\psi|\), de tal forma que o produto interno pode ser
representado por \((\psi,\psi)=\langle\psi|\psi\rangle\).

Note que \(\langle\psi|=|\psi\rangle^*\), corresponde ao complexo
conjugado transposto do vetor de estado \(|\psi\rangle\). Isso fica
claro, quando observamos a representação matricial desse vetores.
Considere, por exemplo, que o vetor de estado tenha \(n\) componentes
(\(c_1,c_2,...,c_n\)). Neste caso, o ``\emph{ket}'' \(|\psi\rangle\) é
escrito como um vetor coluna, enquanto o seu vetor dual ``\emph{bra}'' é
um vetor linha, conforme indicado abaixo:

\[
|\psi\rangle = 
\left[ 
\begin{array} 
c_1\\ c_2\\ ...\\ c_n 
\end{array} 
\right] \, 
\Rightarrow \,\,\,
\langle\psi| =
\left[ 
\begin{array} c_1^* & c_2^* & ... & c_n^* \end{array} 
\right].\]

Nesta representação, todas as propriedades anteriores são equivalentes a
operações sobre matrizes (ou vetores linha/coluna), como, por exemplo,
soma (subtração), multiplicação por escalares e combinações lineares
dessas operações. O produto interno (\emph{``braket''}), como é fácil
perceber, corresponde a uma multiplicação de matrizes, resultando num
escalar:

\[\langle\phi|\psi\rangle = 
\left[ \begin{array}{l} b^*_1 & b^*_2 & ... & b^*_n \end{array} \right]
\left[ \begin{array}{c} c_1 \\ c_2 \\ ... \\ c_n \end{array} \right] 
= \begin{array}{l} b^*_1\,c_1 & b^*_2\,c_2 & ...& b^*_n\,c_n \end{array}
= \sum_{k=1}^n b^*_k\,c_k.\]

\hypertarget{propriedades-do-produto-interno}{%
\subsubsection{Propriedades do produto
interno}\label{propriedades-do-produto-interno}}

Reescrevemos aqui as propriedade dos produto interno, na notação de
Dirac. Para os vetores \(|\psi\rangle\) e \(|\phi\rangle\), pertencentes
ao espaço \(\mathcal{H}\), e os escalares \(\alpha\) e \(\beta\) do
campo complexo \(\mathcal{F}\), as seguintes propriedades são
satisfeitas:

\[\begin{array}{ll}
1.\, &\langle\psi|\phi\rangle = \langle\phi|\psi\rangle ^* \\
2.\, &\langle\psi|(\alpha|\phi\rangle+\beta|\eta\rangle) = 
  \alpha\langle\psi|\phi\rangle + \beta\langle\psi|\eta\rangle \\
3.\, &(\alpha\langle\phi| +\beta\langle\eta|)|\psi\rangle = 
  \alpha^*\langle\phi|\psi\rangle + \beta^*\langle\eta|\psi\rangle \\
4.\, &\langle\psi|\psi\rangle \ge 0 \textrm{   sendo igual só se } 
|\psi \rangle = 0 \end{array}\]

Se \(\langle\psi|\Phi\rangle=0\), os vetores são ortogonais. Os
comprimentos (normas) dos vetores são expressos por:

\textbf{Norma do vetor}:

\[||\psi|| = \sqrt{\langle\psi|\psi\rangle}.\]

\textbf{Vetor normalizado} quando:

\[||\psi|| = \sqrt{\langle\psi|\psi\rangle}=1.\]

\textbf{Vetores ortonormais}:

\[\langle u_j | u_k \rangle = \delta_{jk}\]

\[ \delta_{jk} = \left\{ \begin{array}{c} 1 & \textrm{ se }j=k \\ 0 & \textrm{ caso contrário}\end{array}  \right. \]

    \hypertarget{vetores-de-base}{%
\subsection{5.5 Vetores de base}\label{vetores-de-base}}

O conjunto de vetore
\(\{ |\phi_1\rangle, |\phi_2\rangle, \dots,|\phi_n\rangle \}\) formam
uma base do espaço se eles satisfazem os seguinte critérios:

\begin{quote}
\begin{enumerate}
\def\labelenumi{\arabic{enumi}.}
\item
  É possível escrever qualquer vetor do espaço como uma combinação
  linear única dos vetores \(\{ \phi_i \}\).
\item
  O conjunto
  \(\{ |\phi_1\rangle, |\phi_2\rangle, \dots,|\phi_n\rangle \}\) é
  linearmente indenpendente.
\item
  Satisfaz a relação de completeza.
\end{enumerate}
\end{quote}

\textbf{Condição 1:} Se o conjunto
\(\{ |\phi_1\rangle, |\phi_2\rangle, \dots,|\phi_n\rangle \}\) estende
todo o espaço \(\mathcal{H}\), é possível escrever um vetor
\(|\Psi\rangle\) arbitrário como uma combinção linear dos vetores da
base

\[ |\Psi\rangle = c_1 |\phi_1\rangle + c_2 |\phi_2\rangle + \dots + c_n |\phi_n\rangle = \sum_{i=1}^n c_i |\phi_i\rangle\]

onde os coeficientes da expansão são números complexos dados por

\[ c_i = \langle \phi_i | \Psi \rangle.\]

\textbf{Condição 2:} A conjunto
\(\{ |\phi_1\rangle, |\phi_2\rangle, \dots,|\phi_n\rangle \}\) é dito
linearmente independente quando a equação

\[ a_1 |\phi_1\rangle + a_2 |\phi_2\rangle + \dots + a_n |\phi_n\rangle = 0\]

implica que todos os coeficientes são nulos, \(c_1=c_2=...=c_n=0\). Em
outras palavras, não há nenhuma combinação (não trivial) que produza o
vetor nulo.

\begin{quote}
\textbf{Dimensão do espaço} O número de vetores da base fornece a
dimensão do espaço vetorial.
\end{quote}

\textbf{Condição 3:} Um conjunto ortonormal
\(\{ |\phi_1\rangle, |\phi_2\rangle, \dots,|\phi_n\rangle \}\) constitue
uma base se e somente se satisfaz a \textbf{relação de completeza}

\[ \sum_{i=1}^n |\phi_i\rangle \langle \phi_i| = 1 \]

\hypertarget{procedimento-de-gram-schmidt}{%
\subsubsection{Procedimento de
Gram-Schmidt}\label{procedimento-de-gram-schmidt}}

Se tivermos um conjunto de vetores \(\{ |u_i\rangle \}\) que não é
ortonormal, é possível usar o procedimento de Gram-Schmidt para
construir uma base ortonormal a partir desse conjunto inicial. Para
simplificar o entendimento do processo, consideramos um exemplo com 3
vetores de base (num espaço de dimensção 3).

Começamos selecionando um dos vetores do conjunto \(\{ |u_i\rangle \}\)
e definindo o vetor:

\[ |w_1\rangle = |u_1\rangle  \]

A partir disso, constroi-se sucessivamente os vetores seguintes da base
subtraindo deles as componentes nas direções ortonais àquelas já
construídas. Neste caso, por exemplo, as direções \(|w_2\rangle\) e
\(|w_3\rangle\) são construídas subtraindo as componente na direção de
\(|w_1\rangle\) e \(|w_2\rangle\), conforme:

\[\begin{array}{c}
&&|w_2 \rangle =& |u_2\rangle - \frac{\langle w_1 | u_2 \rangle}{\langle w_1 | w_1 \rangle} |w_1\rangle \\ \\
&&|w_3\rangle =& |u_3\rangle - \frac{\langle w_1 | u_3 \rangle }{\langle w_1 | w_1 \rangle} |w_1\rangle - \frac{\langle w_2 | u_3 \rangle }{\langle w_2 | w_2 \rangle} |w_2\rangle \end{array}\]

Finalmente, para obter um conjunto ortonormal \(\{ |v_i\rangle \}\), nós
podemos normalizar cada um dos vetores \(|w_i\rangle\):

\[|v_1\rangle = \frac{ |w_1 \rangle }{||\langle w_1 | w_1 \rangle||}; \,
|v_2\rangle = \frac{ |w_2 \rangle }{||\langle w_2 | w_2 \rangle}||; \,
|v_3\rangle = \frac{ |w_3 \rangle }{||\langle w_3 | w_3 \rangle||}\]

De forma geral, para um cojunto finito de vetores \(\{u_k\}\), de um
espaço vetorial \(\mathcal{U}\) de dimensão \(d\), pode-se escrever os
vetores ortonormais \(\{v_k\}\) através da construindo:

\[
\left|v_{k+1}\right\rangle \equiv \frac{\left|w_{k+1}\right\rangle-\sum_{i=1}^{k}\left\langle v_{i} | w_{k+1}\right\rangle\left|v_{i}\right\rangle}{\|\left|w_{k+1}\right\rangle-\sum_{i=1}^{k}\left\langle v_{i} | w_{k+1}\right\rangle\left|v_{i}\right\rangle \|}.
\]

    \hypertarget{algebra-de-dirac}{%
\subsubsection{Algebra de Dirac}\label{algebra-de-dirac}}

Vejamos como expressar vetores inteiramente em termos do \emph{kets} da
base e manipular \emph{bras} e \emph{kets} de forma algébrica.

\textbf{Representando um \emph{ket} como \emph{bra}}

Para obter o \emph{bra} correspondente a um dado \emph{ket},
\(| \phi\rangle = \alpha |\psi\rangle\), basta tomar o complexo
conjugado:

\[\langle \phi|  = (\alpha |\psi\rangle)^* = \alpha \langle \psi| \]

podemos também escrever \(|\alpha \psi\rangle = \alpha |\psi\rangle\). O
mesmo pode ser feito para o \emph{bra}, mas deve-se tomar um cuidado
extra, neste caso:

\[ \langle \alpha \psi| = \alpha^* \langle \psi|\]

    \begin{quote}
\textbf{Exercício sugerido}

Suponha que \(\{ |u_1 \rangle, |u_2 \rangle, |u_3 \rangle \}\) seja uma
base ortonormal. Nesta base temos:
\[| \psi \rangle = 2i |u_1 \rangle - 3|u_2 \rangle + i|u_3 \rangle\]
\[ | \phi\rangle = 3 |u_1 \rangle - 2|u_2 \rangle + 4|u_3 \rangle \]

\begin{itemize}
\item
  \begin{enumerate}
  \def\labelenumi{\alph{enumi})}
  \tightlist
  \item
    Ache \(\langle\psi|\) e \(\langle \phi|\).\\
  \end{enumerate}
\item
  \begin{enumerate}
  \def\labelenumi{\alph{enumi})}
  \setcounter{enumi}{1}
  \tightlist
  \item
    Calcule o produto interno \(\langle \phi|\psi\rangle\) e mostre que
    igual seu conjugado.
  \end{enumerate}
\item
  \begin{enumerate}
  \def\labelenumi{\alph{enumi})}
  \setcounter{enumi}{2}
  \tightlist
  \item
    Sendo \(a = 3 + 3i\), calcule \(|a\psi\rangle\).
  \end{enumerate}
\item
  \begin{enumerate}
  \def\labelenumi{\alph{enumi})}
  \setcounter{enumi}{3}
  \tightlist
  \item
    Ache as expressões de \(|\psi+\phi\rangle\) e \(|\psi-\phi\rangle\)
  \end{enumerate}
\item
  \begin{enumerate}
  \def\labelenumi{\alph{enumi})}
  \setcounter{enumi}{4}
  \tightlist
  \item
    Calcule \(\langle a \psi |\) e compare com \(a^* \langle \psi|\).
  \end{enumerate}
\item
  \begin{enumerate}
  \def\labelenumi{\alph{enumi})}
  \setcounter{enumi}{5}
  \tightlist
  \item
    Normalize o vetor \(| \psi \rangle\).
  \end{enumerate}
\end{itemize}
\end{quote}

    \textbf{Encontrando os coeficientes da expansão}

Da mesma forma que fazemos os vetores do espaço Euclidiano, para
encontrar as componentes de um vetor no espaço de Hilber basta fazer o
produto escalar (interno) do vetor com o correspondente verto da base.
Em notação de Dirac, se o vetor é dado por \[
|\psi\rangle=c_{1}\left|u_{1}\right\rangle+c_{2}\left|u_{2}\right\rangle+\cdots+c_{n}\left|u_{n}\right\rangle=\sum_{i=1}^{n} c_{i}\left|u_{i}\right\rangle
\]\\
os coeficientes são dados por \[
c_i = \left\langle\ u_i | \psi \right\rangle
\] que podem ser convenientemente escritos na forma

\[
| \psi \rangle \rightarrow\left( \begin{array}{c} \left\langle u_{1} | \psi\right\rangle \\ \left\langle u_{2} | \psi\right\rangle \\ \vdots \\ \left\langle u_{n} | \psi\right\rangle \end{array} \right) = 
\left(\begin{array}{c} c_{1} \\ c_{2} \\ \vdots \\ c_{n} \end{array}\right)
\]

Note, porém, que um vetor pode ser escrito em termos de diversas bases
diferentes (o vetor tem existência indepentende da base) e em cada uma
delas os valores das componentes serão diferentes.

\textbf{\emph{Exemplo:}} Considere o vetor abaixo, expresso em termos de
uma base ortonormal:
\[ |\psi\rangle=2 i\left|u_{1}\right\rangle-3\left|u_{2}\right\rangle+i\left|u_{3}\right\rangle\]

    Neste caso, o velor coluna dos coeficientes representando
\(|\psi\rangle\) é dado por

\[
|\psi\rangle = 
\left( \begin{array}{c} \left\langle u_{1} | \psi\right\rangle \\ \left\langle u_{2} | \psi\right\rangle \\ \left\langle u_{3} |\psi\right\rangle \end{array}\right) = 
\left( \begin{array}{c} 2 i \\ -3 \\ i \end{array} \right).
\]

Da mesma forma, o vetor dual (``\emph{bra}'') correspondente ao vetor
\(\psi\rangle\) pode ser representado na forma de um vetor linha

\[
\left\langle\psi\left|=\left(\left\langle\psi | u_{1}\right\rangle\left\langle\psi | u_{2}\right\rangle\left\langle\psi | u_{3}\right\rangle\right)=\left(\left\langle u_{1} | \psi\right\rangle^{*}\left\langle u_{2} | \psi\right\rangle^{*}\left\langle u_{3} | \psi\right\rangle^{*}\right)\right.\right.
\]

e portanto

\[
\langle\psi|=\left((2 i)^{*}(-3)^{*}(i)^{*}\right)=(-2 i-3-i).
\]

\hypertarget{operadores-lineares}{%
\subsection{5.6 Operadores lineares}\label{operadores-lineares}}

Grandezas físicas observáveis, que podem ser medidas no laboratório,
como posição e momento, são representandos dentro da estrutura
matemática da mecânica quântica por operadores lineares num espaço
vetorial de Hhilbert. Matematicamente, esses operadores são mapas que
levam (transformam) um vetor em outro vetor. Isto é, são receitas ou
regras de transformação de um dado vetor num novo vetor, geralmente
diferente do primeiro. Frequentemente usa-se como símbolo uma letra
maíscula com ``chapel'' (sinal circunflexo) sobre a letra para indicar
um operador. Assim, na notação de Dirac, escreve-se, por exemplo: \[
\hat{T}|\psi\rangle=|\phi \rangle.
\]

Os operadores que mais nos interessam na MQ são os operadores lineares.
Um operador \(\hat{T}:\mathcal{H}\rightarrow\mathcal{H}\) é linear no
espaço \(\mathcal{H}\) se, dados escalares
\(\alpha, \beta \in \mathbb{C}\) e vetores
\(|u\rangle, |v\rangle \in \mathcal{H}\), ele satisfaz a relação: \[
\hat{T}(\alpha|u\rangle+\beta|v\rangle)=\alpha\, \hat{T}|u\rangle+\beta\, \hat{T}|v\rangle.
\]

Além disso, os operadores lineare também satisfazem as seguintes
relações:

\[
(\hat{T}+\hat{S})\ket{u}=\hat{T}\ket{u} + \hat{S}\ket{u}
\]

\[ 
(\hat{T}\,\hat{S})\ket{u}=\hat{T}(\hat{S}\ket{u})
\]

Operadores atuam tanto nos vetores \emph{kets} como nos vetores duais
\emph{bras}, seguindo a seguinte notação (atenção para a ordem!): \[ 
\hat{T}\ket{u} \quad \text{ ou } \quad \bra{u} \hat{T}
\] mas nunca \((\,\ket{u} \hat{T}\,)\) ou \((\,\hat{T} \bra{u}\,)\), que
são formas incorretas (inválidas)!

\hypertarget{exemplos-importantes}{%
\subsubsection{Exemplos importantes}\label{exemplos-importantes}}

\begin{itemize}
\item
  \textbf{Operador Identidade:} o operador mais simples \[
  \mathbb{1}\ket{u}=\ket{u}
  \]
\item
  \textbf{Produto externo (definição):} o produto externo entre
  \emph{kets} e \emph{bras} é dado por \[
  \ket{\psi}\bra{\phi} = \hat{P}
  \] note que o produto externo resulta num operador e não num escalar!
  Essa construção será muito útil, como veremos adiante.
\item
  \textbf{Operador projetor:} usando o produto externo, podemos calcular
  as projeções de um dado vetor numa base \(\{ u_i \}\), fazendo
\end{itemize}

\[
\begin{array}{ll}
\hat{P}_{u_i} = | u_i \rangle \langle u_i |  &\rightarrow \quad \hat{P}_{u_i} | \chi\rangle = | u_i \rangle (\langle u_i |\chi\rangle) = \beta | u_i \rangle \\
\hat{P}_{u} = \sum_i | u_i \rangle \langle u_i |  &\rightarrow \quad  {P}_{u} | \chi\rangle = \sum_i \,c_i | u_i \rangle = | \chi\rangle
\end{array}
\]

\begin{itemize}
\tightlist
\item
  \textbf{Relação de completeza:} usando os resultados anteriores
  podemos observar que
\end{itemize}

\[
|\psi\rangle=\sum_{i=1}^{n}c_i\left|u_{i}\right\rangle = \sum_{i=1}^{n}\left|u_{i}\right\rangle\left\langle u_{i} | \psi\right\rangle=\left(\sum_{i=1}^{n}\left|u_{i}\right\rangle\left\langle u_{i}\right|\right)|\psi\rangle
\]

\[
\sum_{i=1}^{n}\left|u_{i}\right\rangle\left\langle u_{i}\right| = \mathbb{1}
\]

    \hypertarget{representauxe7uxe3o-de-operadores}{%
\subsubsection{Representação de
operadores}\label{representauxe7uxe3o-de-operadores}}

A operação matemática de transformar um vetor de um espaço vetorial
linear num outro vetor, através da ação de um operador linear, pode ser
representada de várias formas. Uma delas é a representação matricial,
onde os operadores são representados por matrizes quadradas e os vetores
por matrizes linhas e colunas. Neste caso, a transformação linear
torna-se uma mera multiplicação dessas matrizes.

É importante lembrar que, da mesma forma que os vetores do espaço, os
operadores têm existência e significado próprios no espaço vetorial e
sua ação independe da representação ou da base escolhida. Por outro
lado, sua representação matricial, em geral, depende da base escolhida.
Devemos lembrar, porém, que a forma matricial é apenas uma das
representações possíveis de um operador linear.

\textbf{Representação matricial}

A matriz de um operador numa dada base pode ser obtida a partir da ação
do operador em cada vetor da base. Assim, se \(\{ u_i \}\) representa o
conjunto de vetores da base, as componentes do operador \(\hat{T}\)
podem ser obtidas através da operação

\[
T_{i j}=\left\langle u_{i}|\hat{T}| u_{j}\right\rangle.
\]

Em um espaço vetorial de dimensão n, as componentes do operador podem
ser arranjadas na forma de uma matriz quadrada \(n \times n\), onde
\(T_{i j}\) representa o elemento na linha \(i\) e coluna \(j\),
conforme:

\[
\begin{aligned}
\hat{T} \rightarrow\left(T_{i j}\right) &=\left(\begin{array}{cccc}
T_{11} & T_{12} & \dots & T_{1 n} \\
T_{21} & T_{22} & \dots & T_{2 n} \\
\vdots & \vdots & \ddots & \vdots \\
T_{n 1} & T_{n 2} & \dots & T_{n n}
\end{array}\right) \\
&=\left(\begin{array}{cccc}
\left\langle u_{1}|\hat{T}| u_{1}\right\rangle & \left\langle u_{1}|\hat{T}| u_{2}\right\rangle & \dots & \left\langle u_{1}|\hat{T}| u_{n}\right\rangle \\
\left\langle u_{2} \hat{T} | u_{1}\right\rangle & \left\langle u_{2}|\hat{T}| u_{2}\right\rangle & \dots & \left\langle u_{2}|\hat{T}| u_{n}\right\rangle \\
\vdots & \vdots & \ddots & \vdots \\
\left\langle u_{n}|\hat{T}| u_{1}\right\rangle & \left\langle u_{n}|\hat{T}| u_{2}\right\rangle & \dots & \left\langle u_{n}|\hat{T}| u_{n}\right\rangle
\end{array}\right)
\end{aligned}
\]

\begin{quote}
\textbf{\emph{Exercício sugerido}}

Suponha uma base ortonormal
\(\left\{\left|u_{1}\right\rangle,\left|u_{2}\right\rangle,\left|u_{3}\right\rangle\right\}\),
um operador \(\hat{A}\) cuja a ação é dada por: \[
\begin{array}{l}
\hat{A}\left|u_{1}\right\rangle=2\left|u_{1}\right\rangle; \\
\hat{A}\left|u_{2}\right\rangle=3\left|u_{1}\right\rangle-i\left|u_{3}\right\rangle; \\
\hat{A}\left|u_{3}\right\rangle=-\left|u_{2}\right\rangle
\end{array}
\] Escreve a matriz que representa o operador nesta base.
\end{quote}

    \textbf{\emph{Definição} : Traço de um operador}

O traço de um operador \(\hat{T}\), denotado por \(\text{Tr}(\hat{T})\),
é definido como sendo a soma dos elementos na diagonal principal da
matriz que o representa

\[
\text{Tr}(\hat{T})=T_{11}+T_{22}+\ldots+T_{n n}=\sum_{i=1}^{n} T_{i i}.
\]

Alternativamente, o traço também pode ser escrito como:

\[
\text{Tr}(\hat{T})=\left\langle u_{1}|\hat{T}| u_{2}\right\rangle+\left\langle u_{2}|\hat{T}| u_{2}\right\rangle+\ldots+\left\langle u_{n}|\hat{T}| u_{n}\right\rangle=\sum_{i=1}^{n}\left\langle u_{i}|\hat{T}| u_{i}\right\rangle
\]

\begin{quote}
\textbf{Exercício sugerido}

O traço de um operador obedece uma relação cíclica, como indicado \[
\operatorname{Tr}(A B C)=\operatorname{Tr}(B C A)=\operatorname{Tr}(C A B)   
\] Prove isso para o caso de dois operadores \(A\) e \(B\), i.e.~prove
que \(\operatorname{Tr}(AB)=\operatorname{Tr}(B A)\)
\end{quote}

\hypertarget{valores-esperados}{%
\subsubsection{Valores esperados}\label{valores-esperados}}

O valore esperado de um operador com relação a um estado \(\Psi\) é dado
por

\[
\langle\hat{A}\rangle=\langle\Psi|\hat{A}| \Psi\rangle
\]

\begin{quote}
\textbf{Exercício sugerido}

Considere uma partícula no estado \[
|\Psi\rangle=2 i\left|u_{1}\right\rangle-\left|u_{2}\right\rangle+4 i\left|u_{3}\right\rangle
\] e um operador \[
\hat{A}=\left|u_{1}\right\rangle\left\langle u_{1}| -2 i| u_{1}\right\rangle\left\langle u_{2}|+| u_{3}\right\rangle\left\langle u_{3}\right|
\] Considerando que \(\{ |u_i\rangle \}\) é uma base ortonormal, calcule
\(\langle \hat{A} \rangle\) nesse estado.
\end{quote}

    \hypertarget{autovalores-e-autovetores}{%
\subsubsection{Autovalores e
autovetores}\label{autovalores-e-autovetores}}

Quando um operador age sobre um dado vetor e o resultado é o mesmo vetor
multiplicado por um escalar, o vetor é chamado de autovetor e o escalar
de autovalor. Assim, por exemplo, no caso da energia total \[
\hat{H}|\psi_n\rangle = E_n |\psi_n \rangle
\]

    No contexto da mecânica quântica, operadores de observáveis físicos têm
como autovalores o conjunto de todas as possíveis medidas daquela
grandeza física, num dado sistema quântico. Os autovetores de um
operador são autoestados do sistema quântico.

\ldots{}

\hypertarget{conjugauxe7uxe3o-hermitiana}{%
\subsubsection{Conjugação
Hermitiana}\label{conjugauxe7uxe3o-hermitiana}}

\ldots{}

\hypertarget{operadores-hermitianos-e-unituxe1rios}{%
\subsubsection{Operadores Hermitianos e
Unitários}\label{operadores-hermitianos-e-unituxe1rios}}

\ldots{}

\hypertarget{comutadores}{%
\subsubsection{Comutadores}\label{comutadores}}

\ldots{}

\hypertarget{transformauxe7uxf5es-lineares-e-mudanuxe7as-de-base}{%
\subsection{5.7 Transformações lineares e mudanças de
base}\label{transformauxe7uxf5es-lineares-e-mudanuxe7as-de-base}}


    % Add a bibliography block to the postdoc
    
    
    
\end{document}

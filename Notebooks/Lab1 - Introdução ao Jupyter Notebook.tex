\documentclass[11pt]{article}

    \usepackage[breakable]{tcolorbox}
    \usepackage{parskip} % Stop auto-indenting (to mimic markdown behaviour)
    
    \usepackage{iftex}
    \ifPDFTeX
    	\usepackage[T1]{fontenc}
    	\usepackage{mathpazo}
    \else
    	\usepackage{fontspec}
    \fi

    % Basic figure setup, for now with no caption control since it's done
    % automatically by Pandoc (which extracts ![](path) syntax from Markdown).
    \usepackage{graphicx}
    % Maintain compatibility with old templates. Remove in nbconvert 6.0
    \let\Oldincludegraphics\includegraphics
    % Ensure that by default, figures have no caption (until we provide a
    % proper Figure object with a Caption API and a way to capture that
    % in the conversion process - todo).
    \usepackage{caption}
    \DeclareCaptionFormat{nocaption}{}
    \captionsetup{format=nocaption,aboveskip=0pt,belowskip=0pt}

    \usepackage[Export]{adjustbox} % Used to constrain images to a maximum size
    \adjustboxset{max size={0.9\linewidth}{0.9\paperheight}}
    \usepackage{float}
    \floatplacement{figure}{H} % forces figures to be placed at the correct location
    \usepackage{xcolor} % Allow colors to be defined
    \usepackage{enumerate} % Needed for markdown enumerations to work
    \usepackage{geometry} % Used to adjust the document margins
    \usepackage{amsmath} % Equations
    \usepackage{amssymb} % Equations
    \usepackage{textcomp} % defines textquotesingle
    % Hack from http://tex.stackexchange.com/a/47451/13684:
    \AtBeginDocument{%
        \def\PYZsq{\textquotesingle}% Upright quotes in Pygmentized code
    }
    \usepackage{upquote} % Upright quotes for verbatim code
    \usepackage{eurosym} % defines \euro
    \usepackage[mathletters]{ucs} % Extended unicode (utf-8) support
    \usepackage{fancyvrb} % verbatim replacement that allows latex
    \usepackage{grffile} % extends the file name processing of package graphics 
                         % to support a larger range
    \makeatletter % fix for grffile with XeLaTeX
    \def\Gread@@xetex#1{%
      \IfFileExists{"\Gin@base".bb}%
      {\Gread@eps{\Gin@base.bb}}%
      {\Gread@@xetex@aux#1}%
    }
    \makeatother

    % The hyperref package gives us a pdf with properly built
    % internal navigation ('pdf bookmarks' for the table of contents,
    % internal cross-reference links, web links for URLs, etc.)
    \usepackage{hyperref}
    % The default LaTeX title has an obnoxious amount of whitespace. By default,
    % titling removes some of it. It also provides customization options.
    \usepackage{titling}
    \usepackage{longtable} % longtable support required by pandoc >1.10
    \usepackage{booktabs}  % table support for pandoc > 1.12.2
    \usepackage[inline]{enumitem} % IRkernel/repr support (it uses the enumerate* environment)
    \usepackage[normalem]{ulem} % ulem is needed to support strikethroughs (\sout)
                                % normalem makes italics be italics, not underlines
    \usepackage{mathrsfs}
    

    
    % Colors for the hyperref package
    \definecolor{urlcolor}{rgb}{0,.145,.698}
    \definecolor{linkcolor}{rgb}{.71,0.21,0.01}
    \definecolor{citecolor}{rgb}{.12,.54,.11}

    % ANSI colors
    \definecolor{ansi-black}{HTML}{3E424D}
    \definecolor{ansi-black-intense}{HTML}{282C36}
    \definecolor{ansi-red}{HTML}{E75C58}
    \definecolor{ansi-red-intense}{HTML}{B22B31}
    \definecolor{ansi-green}{HTML}{00A250}
    \definecolor{ansi-green-intense}{HTML}{007427}
    \definecolor{ansi-yellow}{HTML}{DDB62B}
    \definecolor{ansi-yellow-intense}{HTML}{B27D12}
    \definecolor{ansi-blue}{HTML}{208FFB}
    \definecolor{ansi-blue-intense}{HTML}{0065CA}
    \definecolor{ansi-magenta}{HTML}{D160C4}
    \definecolor{ansi-magenta-intense}{HTML}{A03196}
    \definecolor{ansi-cyan}{HTML}{60C6C8}
    \definecolor{ansi-cyan-intense}{HTML}{258F8F}
    \definecolor{ansi-white}{HTML}{C5C1B4}
    \definecolor{ansi-white-intense}{HTML}{A1A6B2}
    \definecolor{ansi-default-inverse-fg}{HTML}{FFFFFF}
    \definecolor{ansi-default-inverse-bg}{HTML}{000000}

    % commands and environments needed by pandoc snippets
    % extracted from the output of `pandoc -s`
    \providecommand{\tightlist}{%
      \setlength{\itemsep}{0pt}\setlength{\parskip}{0pt}}
    \DefineVerbatimEnvironment{Highlighting}{Verbatim}{commandchars=\\\{\}}
    % Add ',fontsize=\small' for more characters per line
    \newenvironment{Shaded}{}{}
    \newcommand{\KeywordTok}[1]{\textcolor[rgb]{0.00,0.44,0.13}{\textbf{{#1}}}}
    \newcommand{\DataTypeTok}[1]{\textcolor[rgb]{0.56,0.13,0.00}{{#1}}}
    \newcommand{\DecValTok}[1]{\textcolor[rgb]{0.25,0.63,0.44}{{#1}}}
    \newcommand{\BaseNTok}[1]{\textcolor[rgb]{0.25,0.63,0.44}{{#1}}}
    \newcommand{\FloatTok}[1]{\textcolor[rgb]{0.25,0.63,0.44}{{#1}}}
    \newcommand{\CharTok}[1]{\textcolor[rgb]{0.25,0.44,0.63}{{#1}}}
    \newcommand{\StringTok}[1]{\textcolor[rgb]{0.25,0.44,0.63}{{#1}}}
    \newcommand{\CommentTok}[1]{\textcolor[rgb]{0.38,0.63,0.69}{\textit{{#1}}}}
    \newcommand{\OtherTok}[1]{\textcolor[rgb]{0.00,0.44,0.13}{{#1}}}
    \newcommand{\AlertTok}[1]{\textcolor[rgb]{1.00,0.00,0.00}{\textbf{{#1}}}}
    \newcommand{\FunctionTok}[1]{\textcolor[rgb]{0.02,0.16,0.49}{{#1}}}
    \newcommand{\RegionMarkerTok}[1]{{#1}}
    \newcommand{\ErrorTok}[1]{\textcolor[rgb]{1.00,0.00,0.00}{\textbf{{#1}}}}
    \newcommand{\NormalTok}[1]{{#1}}
    
    % Additional commands for more recent versions of Pandoc
    \newcommand{\ConstantTok}[1]{\textcolor[rgb]{0.53,0.00,0.00}{{#1}}}
    \newcommand{\SpecialCharTok}[1]{\textcolor[rgb]{0.25,0.44,0.63}{{#1}}}
    \newcommand{\VerbatimStringTok}[1]{\textcolor[rgb]{0.25,0.44,0.63}{{#1}}}
    \newcommand{\SpecialStringTok}[1]{\textcolor[rgb]{0.73,0.40,0.53}{{#1}}}
    \newcommand{\ImportTok}[1]{{#1}}
    \newcommand{\DocumentationTok}[1]{\textcolor[rgb]{0.73,0.13,0.13}{\textit{{#1}}}}
    \newcommand{\AnnotationTok}[1]{\textcolor[rgb]{0.38,0.63,0.69}{\textbf{\textit{{#1}}}}}
    \newcommand{\CommentVarTok}[1]{\textcolor[rgb]{0.38,0.63,0.69}{\textbf{\textit{{#1}}}}}
    \newcommand{\VariableTok}[1]{\textcolor[rgb]{0.10,0.09,0.49}{{#1}}}
    \newcommand{\ControlFlowTok}[1]{\textcolor[rgb]{0.00,0.44,0.13}{\textbf{{#1}}}}
    \newcommand{\OperatorTok}[1]{\textcolor[rgb]{0.40,0.40,0.40}{{#1}}}
    \newcommand{\BuiltInTok}[1]{{#1}}
    \newcommand{\ExtensionTok}[1]{{#1}}
    \newcommand{\PreprocessorTok}[1]{\textcolor[rgb]{0.74,0.48,0.00}{{#1}}}
    \newcommand{\AttributeTok}[1]{\textcolor[rgb]{0.49,0.56,0.16}{{#1}}}
    \newcommand{\InformationTok}[1]{\textcolor[rgb]{0.38,0.63,0.69}{\textbf{\textit{{#1}}}}}
    \newcommand{\WarningTok}[1]{\textcolor[rgb]{0.38,0.63,0.69}{\textbf{\textit{{#1}}}}}
    
    
    % Define a nice break command that doesn't care if a line doesn't already
    % exist.
    \def\br{\hspace*{\fill} \\* }
    % Math Jax compatibility definitions
    \def\gt{>}
    \def\lt{<}
    \let\Oldtex\TeX
    \let\Oldlatex\LaTeX
    \renewcommand{\TeX}{\textrm{\Oldtex}}
    \renewcommand{\LaTeX}{\textrm{\Oldlatex}}
    % Document parameters
    % Document title
    \title{Lab1 - Introdução ao Jupyter Notebook}
    
    
    
    
    
% Pygments definitions
\makeatletter
\def\PY@reset{\let\PY@it=\relax \let\PY@bf=\relax%
    \let\PY@ul=\relax \let\PY@tc=\relax%
    \let\PY@bc=\relax \let\PY@ff=\relax}
\def\PY@tok#1{\csname PY@tok@#1\endcsname}
\def\PY@toks#1+{\ifx\relax#1\empty\else%
    \PY@tok{#1}\expandafter\PY@toks\fi}
\def\PY@do#1{\PY@bc{\PY@tc{\PY@ul{%
    \PY@it{\PY@bf{\PY@ff{#1}}}}}}}
\def\PY#1#2{\PY@reset\PY@toks#1+\relax+\PY@do{#2}}

\expandafter\def\csname PY@tok@w\endcsname{\def\PY@tc##1{\textcolor[rgb]{0.73,0.73,0.73}{##1}}}
\expandafter\def\csname PY@tok@c\endcsname{\let\PY@it=\textit\def\PY@tc##1{\textcolor[rgb]{0.25,0.50,0.50}{##1}}}
\expandafter\def\csname PY@tok@cp\endcsname{\def\PY@tc##1{\textcolor[rgb]{0.74,0.48,0.00}{##1}}}
\expandafter\def\csname PY@tok@k\endcsname{\let\PY@bf=\textbf\def\PY@tc##1{\textcolor[rgb]{0.00,0.50,0.00}{##1}}}
\expandafter\def\csname PY@tok@kp\endcsname{\def\PY@tc##1{\textcolor[rgb]{0.00,0.50,0.00}{##1}}}
\expandafter\def\csname PY@tok@kt\endcsname{\def\PY@tc##1{\textcolor[rgb]{0.69,0.00,0.25}{##1}}}
\expandafter\def\csname PY@tok@o\endcsname{\def\PY@tc##1{\textcolor[rgb]{0.40,0.40,0.40}{##1}}}
\expandafter\def\csname PY@tok@ow\endcsname{\let\PY@bf=\textbf\def\PY@tc##1{\textcolor[rgb]{0.67,0.13,1.00}{##1}}}
\expandafter\def\csname PY@tok@nb\endcsname{\def\PY@tc##1{\textcolor[rgb]{0.00,0.50,0.00}{##1}}}
\expandafter\def\csname PY@tok@nf\endcsname{\def\PY@tc##1{\textcolor[rgb]{0.00,0.00,1.00}{##1}}}
\expandafter\def\csname PY@tok@nc\endcsname{\let\PY@bf=\textbf\def\PY@tc##1{\textcolor[rgb]{0.00,0.00,1.00}{##1}}}
\expandafter\def\csname PY@tok@nn\endcsname{\let\PY@bf=\textbf\def\PY@tc##1{\textcolor[rgb]{0.00,0.00,1.00}{##1}}}
\expandafter\def\csname PY@tok@ne\endcsname{\let\PY@bf=\textbf\def\PY@tc##1{\textcolor[rgb]{0.82,0.25,0.23}{##1}}}
\expandafter\def\csname PY@tok@nv\endcsname{\def\PY@tc##1{\textcolor[rgb]{0.10,0.09,0.49}{##1}}}
\expandafter\def\csname PY@tok@no\endcsname{\def\PY@tc##1{\textcolor[rgb]{0.53,0.00,0.00}{##1}}}
\expandafter\def\csname PY@tok@nl\endcsname{\def\PY@tc##1{\textcolor[rgb]{0.63,0.63,0.00}{##1}}}
\expandafter\def\csname PY@tok@ni\endcsname{\let\PY@bf=\textbf\def\PY@tc##1{\textcolor[rgb]{0.60,0.60,0.60}{##1}}}
\expandafter\def\csname PY@tok@na\endcsname{\def\PY@tc##1{\textcolor[rgb]{0.49,0.56,0.16}{##1}}}
\expandafter\def\csname PY@tok@nt\endcsname{\let\PY@bf=\textbf\def\PY@tc##1{\textcolor[rgb]{0.00,0.50,0.00}{##1}}}
\expandafter\def\csname PY@tok@nd\endcsname{\def\PY@tc##1{\textcolor[rgb]{0.67,0.13,1.00}{##1}}}
\expandafter\def\csname PY@tok@s\endcsname{\def\PY@tc##1{\textcolor[rgb]{0.73,0.13,0.13}{##1}}}
\expandafter\def\csname PY@tok@sd\endcsname{\let\PY@it=\textit\def\PY@tc##1{\textcolor[rgb]{0.73,0.13,0.13}{##1}}}
\expandafter\def\csname PY@tok@si\endcsname{\let\PY@bf=\textbf\def\PY@tc##1{\textcolor[rgb]{0.73,0.40,0.53}{##1}}}
\expandafter\def\csname PY@tok@se\endcsname{\let\PY@bf=\textbf\def\PY@tc##1{\textcolor[rgb]{0.73,0.40,0.13}{##1}}}
\expandafter\def\csname PY@tok@sr\endcsname{\def\PY@tc##1{\textcolor[rgb]{0.73,0.40,0.53}{##1}}}
\expandafter\def\csname PY@tok@ss\endcsname{\def\PY@tc##1{\textcolor[rgb]{0.10,0.09,0.49}{##1}}}
\expandafter\def\csname PY@tok@sx\endcsname{\def\PY@tc##1{\textcolor[rgb]{0.00,0.50,0.00}{##1}}}
\expandafter\def\csname PY@tok@m\endcsname{\def\PY@tc##1{\textcolor[rgb]{0.40,0.40,0.40}{##1}}}
\expandafter\def\csname PY@tok@gh\endcsname{\let\PY@bf=\textbf\def\PY@tc##1{\textcolor[rgb]{0.00,0.00,0.50}{##1}}}
\expandafter\def\csname PY@tok@gu\endcsname{\let\PY@bf=\textbf\def\PY@tc##1{\textcolor[rgb]{0.50,0.00,0.50}{##1}}}
\expandafter\def\csname PY@tok@gd\endcsname{\def\PY@tc##1{\textcolor[rgb]{0.63,0.00,0.00}{##1}}}
\expandafter\def\csname PY@tok@gi\endcsname{\def\PY@tc##1{\textcolor[rgb]{0.00,0.63,0.00}{##1}}}
\expandafter\def\csname PY@tok@gr\endcsname{\def\PY@tc##1{\textcolor[rgb]{1.00,0.00,0.00}{##1}}}
\expandafter\def\csname PY@tok@ge\endcsname{\let\PY@it=\textit}
\expandafter\def\csname PY@tok@gs\endcsname{\let\PY@bf=\textbf}
\expandafter\def\csname PY@tok@gp\endcsname{\let\PY@bf=\textbf\def\PY@tc##1{\textcolor[rgb]{0.00,0.00,0.50}{##1}}}
\expandafter\def\csname PY@tok@go\endcsname{\def\PY@tc##1{\textcolor[rgb]{0.53,0.53,0.53}{##1}}}
\expandafter\def\csname PY@tok@gt\endcsname{\def\PY@tc##1{\textcolor[rgb]{0.00,0.27,0.87}{##1}}}
\expandafter\def\csname PY@tok@err\endcsname{\def\PY@bc##1{\setlength{\fboxsep}{0pt}\fcolorbox[rgb]{1.00,0.00,0.00}{1,1,1}{\strut ##1}}}
\expandafter\def\csname PY@tok@kc\endcsname{\let\PY@bf=\textbf\def\PY@tc##1{\textcolor[rgb]{0.00,0.50,0.00}{##1}}}
\expandafter\def\csname PY@tok@kd\endcsname{\let\PY@bf=\textbf\def\PY@tc##1{\textcolor[rgb]{0.00,0.50,0.00}{##1}}}
\expandafter\def\csname PY@tok@kn\endcsname{\let\PY@bf=\textbf\def\PY@tc##1{\textcolor[rgb]{0.00,0.50,0.00}{##1}}}
\expandafter\def\csname PY@tok@kr\endcsname{\let\PY@bf=\textbf\def\PY@tc##1{\textcolor[rgb]{0.00,0.50,0.00}{##1}}}
\expandafter\def\csname PY@tok@bp\endcsname{\def\PY@tc##1{\textcolor[rgb]{0.00,0.50,0.00}{##1}}}
\expandafter\def\csname PY@tok@fm\endcsname{\def\PY@tc##1{\textcolor[rgb]{0.00,0.00,1.00}{##1}}}
\expandafter\def\csname PY@tok@vc\endcsname{\def\PY@tc##1{\textcolor[rgb]{0.10,0.09,0.49}{##1}}}
\expandafter\def\csname PY@tok@vg\endcsname{\def\PY@tc##1{\textcolor[rgb]{0.10,0.09,0.49}{##1}}}
\expandafter\def\csname PY@tok@vi\endcsname{\def\PY@tc##1{\textcolor[rgb]{0.10,0.09,0.49}{##1}}}
\expandafter\def\csname PY@tok@vm\endcsname{\def\PY@tc##1{\textcolor[rgb]{0.10,0.09,0.49}{##1}}}
\expandafter\def\csname PY@tok@sa\endcsname{\def\PY@tc##1{\textcolor[rgb]{0.73,0.13,0.13}{##1}}}
\expandafter\def\csname PY@tok@sb\endcsname{\def\PY@tc##1{\textcolor[rgb]{0.73,0.13,0.13}{##1}}}
\expandafter\def\csname PY@tok@sc\endcsname{\def\PY@tc##1{\textcolor[rgb]{0.73,0.13,0.13}{##1}}}
\expandafter\def\csname PY@tok@dl\endcsname{\def\PY@tc##1{\textcolor[rgb]{0.73,0.13,0.13}{##1}}}
\expandafter\def\csname PY@tok@s2\endcsname{\def\PY@tc##1{\textcolor[rgb]{0.73,0.13,0.13}{##1}}}
\expandafter\def\csname PY@tok@sh\endcsname{\def\PY@tc##1{\textcolor[rgb]{0.73,0.13,0.13}{##1}}}
\expandafter\def\csname PY@tok@s1\endcsname{\def\PY@tc##1{\textcolor[rgb]{0.73,0.13,0.13}{##1}}}
\expandafter\def\csname PY@tok@mb\endcsname{\def\PY@tc##1{\textcolor[rgb]{0.40,0.40,0.40}{##1}}}
\expandafter\def\csname PY@tok@mf\endcsname{\def\PY@tc##1{\textcolor[rgb]{0.40,0.40,0.40}{##1}}}
\expandafter\def\csname PY@tok@mh\endcsname{\def\PY@tc##1{\textcolor[rgb]{0.40,0.40,0.40}{##1}}}
\expandafter\def\csname PY@tok@mi\endcsname{\def\PY@tc##1{\textcolor[rgb]{0.40,0.40,0.40}{##1}}}
\expandafter\def\csname PY@tok@il\endcsname{\def\PY@tc##1{\textcolor[rgb]{0.40,0.40,0.40}{##1}}}
\expandafter\def\csname PY@tok@mo\endcsname{\def\PY@tc##1{\textcolor[rgb]{0.40,0.40,0.40}{##1}}}
\expandafter\def\csname PY@tok@ch\endcsname{\let\PY@it=\textit\def\PY@tc##1{\textcolor[rgb]{0.25,0.50,0.50}{##1}}}
\expandafter\def\csname PY@tok@cm\endcsname{\let\PY@it=\textit\def\PY@tc##1{\textcolor[rgb]{0.25,0.50,0.50}{##1}}}
\expandafter\def\csname PY@tok@cpf\endcsname{\let\PY@it=\textit\def\PY@tc##1{\textcolor[rgb]{0.25,0.50,0.50}{##1}}}
\expandafter\def\csname PY@tok@c1\endcsname{\let\PY@it=\textit\def\PY@tc##1{\textcolor[rgb]{0.25,0.50,0.50}{##1}}}
\expandafter\def\csname PY@tok@cs\endcsname{\let\PY@it=\textit\def\PY@tc##1{\textcolor[rgb]{0.25,0.50,0.50}{##1}}}

\def\PYZbs{\char`\\}
\def\PYZus{\char`\_}
\def\PYZob{\char`\{}
\def\PYZcb{\char`\}}
\def\PYZca{\char`\^}
\def\PYZam{\char`\&}
\def\PYZlt{\char`\<}
\def\PYZgt{\char`\>}
\def\PYZsh{\char`\#}
\def\PYZpc{\char`\%}
\def\PYZdl{\char`\$}
\def\PYZhy{\char`\-}
\def\PYZsq{\char`\'}
\def\PYZdq{\char`\"}
\def\PYZti{\char`\~}
% for compatibility with earlier versions
\def\PYZat{@}
\def\PYZlb{[}
\def\PYZrb{]}
\makeatother


    % For linebreaks inside Verbatim environment from package fancyvrb. 
    \makeatletter
        \newbox\Wrappedcontinuationbox 
        \newbox\Wrappedvisiblespacebox 
        \newcommand*\Wrappedvisiblespace {\textcolor{red}{\textvisiblespace}} 
        \newcommand*\Wrappedcontinuationsymbol {\textcolor{red}{\llap{\tiny$\m@th\hookrightarrow$}}} 
        \newcommand*\Wrappedcontinuationindent {3ex } 
        \newcommand*\Wrappedafterbreak {\kern\Wrappedcontinuationindent\copy\Wrappedcontinuationbox} 
        % Take advantage of the already applied Pygments mark-up to insert 
        % potential linebreaks for TeX processing. 
        %        {, <, #, %, $, ' and ": go to next line. 
        %        _, }, ^, &, >, - and ~: stay at end of broken line. 
        % Use of \textquotesingle for straight quote. 
        \newcommand*\Wrappedbreaksatspecials {% 
            \def\PYGZus{\discretionary{\char`\_}{\Wrappedafterbreak}{\char`\_}}% 
            \def\PYGZob{\discretionary{}{\Wrappedafterbreak\char`\{}{\char`\{}}% 
            \def\PYGZcb{\discretionary{\char`\}}{\Wrappedafterbreak}{\char`\}}}% 
            \def\PYGZca{\discretionary{\char`\^}{\Wrappedafterbreak}{\char`\^}}% 
            \def\PYGZam{\discretionary{\char`\&}{\Wrappedafterbreak}{\char`\&}}% 
            \def\PYGZlt{\discretionary{}{\Wrappedafterbreak\char`\<}{\char`\<}}% 
            \def\PYGZgt{\discretionary{\char`\>}{\Wrappedafterbreak}{\char`\>}}% 
            \def\PYGZsh{\discretionary{}{\Wrappedafterbreak\char`\#}{\char`\#}}% 
            \def\PYGZpc{\discretionary{}{\Wrappedafterbreak\char`\%}{\char`\%}}% 
            \def\PYGZdl{\discretionary{}{\Wrappedafterbreak\char`\$}{\char`\$}}% 
            \def\PYGZhy{\discretionary{\char`\-}{\Wrappedafterbreak}{\char`\-}}% 
            \def\PYGZsq{\discretionary{}{\Wrappedafterbreak\textquotesingle}{\textquotesingle}}% 
            \def\PYGZdq{\discretionary{}{\Wrappedafterbreak\char`\"}{\char`\"}}% 
            \def\PYGZti{\discretionary{\char`\~}{\Wrappedafterbreak}{\char`\~}}% 
        } 
        % Some characters . , ; ? ! / are not pygmentized. 
        % This macro makes them "active" and they will insert potential linebreaks 
        \newcommand*\Wrappedbreaksatpunct {% 
            \lccode`\~`\.\lowercase{\def~}{\discretionary{\hbox{\char`\.}}{\Wrappedafterbreak}{\hbox{\char`\.}}}% 
            \lccode`\~`\,\lowercase{\def~}{\discretionary{\hbox{\char`\,}}{\Wrappedafterbreak}{\hbox{\char`\,}}}% 
            \lccode`\~`\;\lowercase{\def~}{\discretionary{\hbox{\char`\;}}{\Wrappedafterbreak}{\hbox{\char`\;}}}% 
            \lccode`\~`\:\lowercase{\def~}{\discretionary{\hbox{\char`\:}}{\Wrappedafterbreak}{\hbox{\char`\:}}}% 
            \lccode`\~`\?\lowercase{\def~}{\discretionary{\hbox{\char`\?}}{\Wrappedafterbreak}{\hbox{\char`\?}}}% 
            \lccode`\~`\!\lowercase{\def~}{\discretionary{\hbox{\char`\!}}{\Wrappedafterbreak}{\hbox{\char`\!}}}% 
            \lccode`\~`\/\lowercase{\def~}{\discretionary{\hbox{\char`\/}}{\Wrappedafterbreak}{\hbox{\char`\/}}}% 
            \catcode`\.\active
            \catcode`\,\active 
            \catcode`\;\active
            \catcode`\:\active
            \catcode`\?\active
            \catcode`\!\active
            \catcode`\/\active 
            \lccode`\~`\~ 	
        }
    \makeatother

    \let\OriginalVerbatim=\Verbatim
    \makeatletter
    \renewcommand{\Verbatim}[1][1]{%
        %\parskip\z@skip
        \sbox\Wrappedcontinuationbox {\Wrappedcontinuationsymbol}%
        \sbox\Wrappedvisiblespacebox {\FV@SetupFont\Wrappedvisiblespace}%
        \def\FancyVerbFormatLine ##1{\hsize\linewidth
            \vtop{\raggedright\hyphenpenalty\z@\exhyphenpenalty\z@
                \doublehyphendemerits\z@\finalhyphendemerits\z@
                \strut ##1\strut}%
        }%
        % If the linebreak is at a space, the latter will be displayed as visible
        % space at end of first line, and a continuation symbol starts next line.
        % Stretch/shrink are however usually zero for typewriter font.
        \def\FV@Space {%
            \nobreak\hskip\z@ plus\fontdimen3\font minus\fontdimen4\font
            \discretionary{\copy\Wrappedvisiblespacebox}{\Wrappedafterbreak}
            {\kern\fontdimen2\font}%
        }%
        
        % Allow breaks at special characters using \PYG... macros.
        \Wrappedbreaksatspecials
        % Breaks at punctuation characters . , ; ? ! and / need catcode=\active 	
        \OriginalVerbatim[#1,codes*=\Wrappedbreaksatpunct]%
    }
    \makeatother

    % Exact colors from NB
    \definecolor{incolor}{HTML}{303F9F}
    \definecolor{outcolor}{HTML}{D84315}
    \definecolor{cellborder}{HTML}{CFCFCF}
    \definecolor{cellbackground}{HTML}{F7F7F7}
    
    % prompt
    \makeatletter
    \newcommand{\boxspacing}{\kern\kvtcb@left@rule\kern\kvtcb@boxsep}
    \makeatother
    \newcommand{\prompt}[4]{
        \ttfamily\llap{{\color{#2}[#3]:\hspace{3pt}#4}}\vspace{-\baselineskip}
    }
    

    
    % Prevent overflowing lines due to hard-to-break entities
    \sloppy 
    % Setup hyperref package
    \hypersetup{
      breaklinks=true,  % so long urls are correctly broken across lines
      colorlinks=true,
      urlcolor=urlcolor,
      linkcolor=linkcolor,
      citecolor=citecolor,
      }
    % Slightly bigger margins than the latex defaults
    
    \geometry{verbose,tmargin=1in,bmargin=1in,lmargin=1in,rmargin=1in}
    
    

\begin{document}
    
    \maketitle
    
    

    
    \hypertarget{lab-1-introduuxe7uxe3o-ao-jupyter-notebook}{%
\section{Lab 1: Introdução ao Jupyter
notebook}\label{lab-1-introduuxe7uxe3o-ao-jupyter-notebook}}

Neste notebook veremos uma breve introdução ao ambiente
\href{https://jupyter.org/}{Jupyter} e a linguagem
\href{https://docs.python.org/pt-br/3.7/tutorial/index.html}{Python}.
Veremos também como utilizar alguns elementos básicos de interesse do
nosso curso. Como, por exemplo, expressar números complexos, vetores e
matrizes, além de escrever e usar funções matemáticas e produzir
gráficos em Python.

Na próxima aula prática, veremos como usar ferramentas de computação
algébrica, com expressões analíticas.

    \hypertarget{notebook-basics}{%
\subsection{Notebook Basics}\label{notebook-basics}}

O \emph{Jupyter notebook} permite reunir num mesmo documento um ambiente
interativo com \textbf{textos} e \textbf{fórmulas matemáticas},
\textbf{figuras} e \textbf{gráficos}, \textbf{código em Python} (ou
outras linguagens), \textbf{código HTML} e \textbf{widgets interativos},
tudo num só lugar, interagindo de forma integral. É ótimo tanto para
testar e explorar ideias, como comunicá-las a outras pessoas.

Para ter informaçãos de como funciona o ambiente do notebook Jupyter,
acesse os links:

\begin{itemize}
\tightlist
\item
  \href{https://nbviewer.jupyter.org/github/jupyter/notebook/blob/master/docs/source/examples/Notebook/What\%20is\%20the\%20Jupyter\%20Notebook.ipynb}{What
  is Jupyter Notebook}
\item
  \href{https://nbviewer.jupyter.org/github/jupyter/notebook/blob/master/docs/source/examples/Notebook/Notebook\%20Basics.ipynb}{Notebook
  basics}
\item
  \href{https://nbviewer.jupyter.org/github/jupyter/notebook/blob/master/docs/source/examples/Notebook/Working\%20With\%20Markdown\%20Cells.ipynb}{Working
  with Markdown cells}
\item
  \href{https://nbviewer.jupyter.org/github/jupyter/notebook/blob/master/docs/source/examples/Notebook/Typesetting\%20Equations.ipynb}{Typesetting
  Equations (with LaTeX)}
\end{itemize}

Para facilitar, ainda mais e também permitir você explorar/modificar os
códigos fontes, eu incluí uma cópia do diretórios de exemplos aqui, na
sua área neste servidor. Basta procurar o diretório
\textbf{Jupyter\_Tutorial/examples/Notebook} no diretório principal,
quando você faz o \emph{login} neste servidor. Alternativamente, poderá
acessar esse exemplos no
\href{https://github.com/jupyter/notebook/tree/master/docs/source/examples/Notebook}{GitHub}.

Não deixe de acessar e explorar o link acima (além de acessar o código
neste servidor, você pode baixar o \emph{notebook} e explorar o código
localmente, para entender seus elementos -- é uma ótima forma de
aprender rapidamente!). Para acessar o Jupyter (Python) no seu
computador (localmente), eu recomendo instalar a distribuição
\href{https://www.anaconda.com/products/individual}{Anaconda}.

A seguir nós iremos nos concentra em como usar esse ambiente para as
atividades e tarefas desta disciplina.

    \hypertarget{nuxfameros-e-operauxe7uxf5es-buxe1sicas}{%
\subsection{Números e operações
básicas}\label{nuxfameros-e-operauxe7uxf5es-buxe1sicas}}

Python é uma linguagem interpretada, em contraste com liguagens
compiladas, o que significa que, além de gerar programas (scripts) você
também pode interagir diretamente com o interpretador e, por exemplo,
fazer cáculos como numa calculadora.

    \begin{tcolorbox}[breakable, size=fbox, boxrule=1pt, pad at break*=1mm,colback=cellbackground, colframe=cellborder]
\prompt{In}{incolor}{1}{\boxspacing}
\begin{Verbatim}[commandchars=\\\{\}]
\PY{l+m+mi}{3}\PY{o}{+}\PY{l+m+mi}{4}
\end{Verbatim}
\end{tcolorbox}

            \begin{tcolorbox}[breakable, size=fbox, boxrule=.5pt, pad at break*=1mm, opacityfill=0]
\prompt{Out}{outcolor}{1}{\boxspacing}
\begin{Verbatim}[commandchars=\\\{\}]
7
\end{Verbatim}
\end{tcolorbox}
        
    \begin{tcolorbox}[breakable, size=fbox, boxrule=1pt, pad at break*=1mm,colback=cellbackground, colframe=cellborder]
\prompt{In}{incolor}{2}{\boxspacing}
\begin{Verbatim}[commandchars=\\\{\}]
\PY{l+m+mi}{3}\PY{o}{*}\PY{l+m+mi}{7}
\end{Verbatim}
\end{tcolorbox}

            \begin{tcolorbox}[breakable, size=fbox, boxrule=.5pt, pad at break*=1mm, opacityfill=0]
\prompt{Out}{outcolor}{2}{\boxspacing}
\begin{Verbatim}[commandchars=\\\{\}]
21
\end{Verbatim}
\end{tcolorbox}
        
    \begin{tcolorbox}[breakable, size=fbox, boxrule=1pt, pad at break*=1mm,colback=cellbackground, colframe=cellborder]
\prompt{In}{incolor}{3}{\boxspacing}
\begin{Verbatim}[commandchars=\\\{\}]
\PY{l+m+mi}{7}\PY{o}{/}\PY{l+m+mi}{3}
\end{Verbatim}
\end{tcolorbox}

            \begin{tcolorbox}[breakable, size=fbox, boxrule=.5pt, pad at break*=1mm, opacityfill=0]
\prompt{Out}{outcolor}{3}{\boxspacing}
\begin{Verbatim}[commandchars=\\\{\}]
2.3333333333333335
\end{Verbatim}
\end{tcolorbox}
        
    \begin{tcolorbox}[breakable, size=fbox, boxrule=1pt, pad at break*=1mm,colback=cellbackground, colframe=cellborder]
\prompt{In}{incolor}{4}{\boxspacing}
\begin{Verbatim}[commandchars=\\\{\}]
\PY{l+m+mi}{7}\PY{o}{/}\PY{o}{/}\PY{l+m+mi}{3}  \PY{c+c1}{\PYZsh{}divisão inteira}
\end{Verbatim}
\end{tcolorbox}

            \begin{tcolorbox}[breakable, size=fbox, boxrule=.5pt, pad at break*=1mm, opacityfill=0]
\prompt{Out}{outcolor}{4}{\boxspacing}
\begin{Verbatim}[commandchars=\\\{\}]
2
\end{Verbatim}
\end{tcolorbox}
        
    \hypertarget{nuxfameros-complexos}{%
\subsubsection{Números complexos}\label{nuxfameros-complexos}}

    \begin{tcolorbox}[breakable, size=fbox, boxrule=1pt, pad at break*=1mm,colback=cellbackground, colframe=cellborder]
\prompt{In}{incolor}{5}{\boxspacing}
\begin{Verbatim}[commandchars=\\\{\}]
 \PY{l+m+mi}{3} \PY{o}{+} \PY{l+m+mi}{4}\PY{n}{j}
\end{Verbatim}
\end{tcolorbox}

            \begin{tcolorbox}[breakable, size=fbox, boxrule=.5pt, pad at break*=1mm, opacityfill=0]
\prompt{Out}{outcolor}{5}{\boxspacing}
\begin{Verbatim}[commandchars=\\\{\}]
(3+4j)
\end{Verbatim}
\end{tcolorbox}
        
    \begin{tcolorbox}[breakable, size=fbox, boxrule=1pt, pad at break*=1mm,colback=cellbackground, colframe=cellborder]
\prompt{In}{incolor}{6}{\boxspacing}
\begin{Verbatim}[commandchars=\\\{\}]
\PY{p}{(}\PY{l+m+mi}{3}\PY{o}{+}\PY{l+m+mi}{4}\PY{n}{j}\PY{p}{)}\PY{o}{+}\PY{p}{(}\PY{l+m+mi}{7}\PY{o}{+}\PY{l+m+mi}{3}\PY{n}{j}\PY{p}{)}
\end{Verbatim}
\end{tcolorbox}

            \begin{tcolorbox}[breakable, size=fbox, boxrule=.5pt, pad at break*=1mm, opacityfill=0]
\prompt{Out}{outcolor}{6}{\boxspacing}
\begin{Verbatim}[commandchars=\\\{\}]
(10+7j)
\end{Verbatim}
\end{tcolorbox}
        
    \begin{tcolorbox}[breakable, size=fbox, boxrule=1pt, pad at break*=1mm,colback=cellbackground, colframe=cellborder]
\prompt{In}{incolor}{7}{\boxspacing}
\begin{Verbatim}[commandchars=\\\{\}]
\PY{p}{(}\PY{l+m+mi}{3}\PY{o}{+}\PY{l+m+mi}{4}\PY{n}{j}\PY{p}{)}\PY{o}{*}\PY{p}{(}\PY{l+m+mi}{7}\PY{o}{+}\PY{l+m+mi}{3}\PY{n}{j}\PY{p}{)}\PY{o}{+}\PY{l+m+mi}{511}
\end{Verbatim}
\end{tcolorbox}

            \begin{tcolorbox}[breakable, size=fbox, boxrule=.5pt, pad at break*=1mm, opacityfill=0]
\prompt{Out}{outcolor}{7}{\boxspacing}
\begin{Verbatim}[commandchars=\\\{\}]
(520+37j)
\end{Verbatim}
\end{tcolorbox}
        
    Podemos definir variáveis, atribuindo valores numéricos (tanto reais,
como complexos) a essas variáveis

    \begin{tcolorbox}[breakable, size=fbox, boxrule=1pt, pad at break*=1mm,colback=cellbackground, colframe=cellborder]
\prompt{In}{incolor}{8}{\boxspacing}
\begin{Verbatim}[commandchars=\\\{\}]
\PY{n}{r} \PY{o}{=} \PY{l+m+mi}{7}
\PY{n}{z} \PY{o}{=} \PY{l+m+mi}{3} \PY{o}{+} \PY{l+m+mi}{4}\PY{n}{j}
\PY{n}{z}
\end{Verbatim}
\end{tcolorbox}

            \begin{tcolorbox}[breakable, size=fbox, boxrule=.5pt, pad at break*=1mm, opacityfill=0]
\prompt{Out}{outcolor}{8}{\boxspacing}
\begin{Verbatim}[commandchars=\\\{\}]
(3+4j)
\end{Verbatim}
\end{tcolorbox}
        
    Usando o método \textbf{.conjugate()} é possível acessar o complexo
cojugado do valor associado a uma variável.

    \begin{tcolorbox}[breakable, size=fbox, boxrule=1pt, pad at break*=1mm,colback=cellbackground, colframe=cellborder]
\prompt{In}{incolor}{9}{\boxspacing}
\begin{Verbatim}[commandchars=\\\{\}]
\PY{n}{z}\PY{o}{.}\PY{n}{conjugate}\PY{p}{(}\PY{p}{)}
\end{Verbatim}
\end{tcolorbox}

            \begin{tcolorbox}[breakable, size=fbox, boxrule=.5pt, pad at break*=1mm, opacityfill=0]
\prompt{Out}{outcolor}{9}{\boxspacing}
\begin{Verbatim}[commandchars=\\\{\}]
(3-4j)
\end{Verbatim}
\end{tcolorbox}
        
    \begin{tcolorbox}[breakable, size=fbox, boxrule=1pt, pad at break*=1mm,colback=cellbackground, colframe=cellborder]
\prompt{In}{incolor}{10}{\boxspacing}
\begin{Verbatim}[commandchars=\\\{\}]
\PY{n}{r}\PY{o}{.}\PY{n}{conjugate}\PY{p}{(}\PY{p}{)}
\end{Verbatim}
\end{tcolorbox}

            \begin{tcolorbox}[breakable, size=fbox, boxrule=.5pt, pad at break*=1mm, opacityfill=0]
\prompt{Out}{outcolor}{10}{\boxspacing}
\begin{Verbatim}[commandchars=\\\{\}]
7
\end{Verbatim}
\end{tcolorbox}
        
    (😃) \textbf{Variáveis dinâmicas:} em Python a atribuição de variáveis é
feita dinamicamente, no sentido que o tipo da variável (i.e., como ela é
armazenada internamente na memória) é determinado automaticamente,
dependo do valor atribuído a ela. Por isso não é necessário escolher
\emph{a priori} o tipo da variável. É só usar\ldots{} ( \emph{mas tome
cuidado!} )

    \hypertarget{listas}{%
\subsubsection{Listas}\label{listas}}

    \begin{tcolorbox}[breakable, size=fbox, boxrule=1pt, pad at break*=1mm,colback=cellbackground, colframe=cellborder]
\prompt{In}{incolor}{11}{\boxspacing}
\begin{Verbatim}[commandchars=\\\{\}]
\PY{n}{lista} \PY{o}{=} \PY{p}{[}\PY{l+m+mi}{1}\PY{p}{,} \PY{l+m+mi}{4}\PY{p}{,} \PY{l+m+mi}{9}\PY{p}{,} \PY{l+m+mi}{16}\PY{p}{,} \PY{l+m+mi}{25}\PY{p}{]}  \PY{c+c1}{\PYZsh{} atribui a lista a uma variável}
\PY{n}{lista}  \PY{c+c1}{\PYZsh{}mostra os resultados (valor da variável)}
\end{Verbatim}
\end{tcolorbox}

            \begin{tcolorbox}[breakable, size=fbox, boxrule=.5pt, pad at break*=1mm, opacityfill=0]
\prompt{Out}{outcolor}{11}{\boxspacing}
\begin{Verbatim}[commandchars=\\\{\}]
[1, 4, 9, 16, 25]
\end{Verbatim}
\end{tcolorbox}
        
    Você pode acessar e ``manipular'' os valores dos elementos na lista,
usando seu índice. Os índices em Python sempre começam no elemento
``zero''\ldots{}

    \begin{tcolorbox}[breakable, size=fbox, boxrule=1pt, pad at break*=1mm,colback=cellbackground, colframe=cellborder]
\prompt{In}{incolor}{12}{\boxspacing}
\begin{Verbatim}[commandchars=\\\{\}]
\PY{n}{lista}\PY{p}{[}\PY{l+m+mi}{2}\PY{p}{]}  \PY{c+c1}{\PYZsh{} acessa o terceiro elemento da lista}
\end{Verbatim}
\end{tcolorbox}

            \begin{tcolorbox}[breakable, size=fbox, boxrule=.5pt, pad at break*=1mm, opacityfill=0]
\prompt{Out}{outcolor}{12}{\boxspacing}
\begin{Verbatim}[commandchars=\\\{\}]
9
\end{Verbatim}
\end{tcolorbox}
        
    \begin{tcolorbox}[breakable, size=fbox, boxrule=1pt, pad at break*=1mm,colback=cellbackground, colframe=cellborder]
\prompt{In}{incolor}{13}{\boxspacing}
\begin{Verbatim}[commandchars=\\\{\}]
\PY{n}{lista}\PY{p}{[}\PY{l+m+mi}{0}\PY{p}{]}  \PY{c+c1}{\PYZsh{} este é o primeiro elemento}
\end{Verbatim}
\end{tcolorbox}

            \begin{tcolorbox}[breakable, size=fbox, boxrule=.5pt, pad at break*=1mm, opacityfill=0]
\prompt{Out}{outcolor}{13}{\boxspacing}
\begin{Verbatim}[commandchars=\\\{\}]
1
\end{Verbatim}
\end{tcolorbox}
        
    \begin{tcolorbox}[breakable, size=fbox, boxrule=1pt, pad at break*=1mm,colback=cellbackground, colframe=cellborder]
\prompt{In}{incolor}{14}{\boxspacing}
\begin{Verbatim}[commandchars=\\\{\}]
\PY{n}{lista}\PY{p}{[}\PY{o}{\PYZhy{}}\PY{l+m+mi}{1}\PY{p}{]}  \PY{c+c1}{\PYZsh{} e este, consegue imaginar?? }
\end{Verbatim}
\end{tcolorbox}

            \begin{tcolorbox}[breakable, size=fbox, boxrule=.5pt, pad at break*=1mm, opacityfill=0]
\prompt{Out}{outcolor}{14}{\boxspacing}
\begin{Verbatim}[commandchars=\\\{\}]
25
\end{Verbatim}
\end{tcolorbox}
        
    Listas podem ter diferentes tipos de elementos, não apenas números:

    \begin{tcolorbox}[breakable, size=fbox, boxrule=1pt, pad at break*=1mm,colback=cellbackground, colframe=cellborder]
\prompt{In}{incolor}{15}{\boxspacing}
\begin{Verbatim}[commandchars=\\\{\}]
\PY{n}{lista2} \PY{o}{=} \PY{p}{[}\PY{l+s+s1}{\PYZsq{}}\PY{l+s+s1}{a}\PY{l+s+s1}{\PYZsq{}}\PY{p}{,} \PY{l+s+s1}{\PYZsq{}}\PY{l+s+s1}{b}\PY{l+s+s1}{\PYZsq{}}\PY{p}{,} \PY{l+s+s1}{\PYZsq{}}\PY{l+s+s1}{c}\PY{l+s+s1}{\PYZsq{}}\PY{p}{,} \PY{l+s+s1}{\PYZsq{}}\PY{l+s+s1}{d}\PY{l+s+s1}{\PYZsq{}}\PY{p}{,}\PY{l+s+s1}{\PYZsq{}}\PY{l+s+s1}{texto também pode!}\PY{l+s+s1}{\PYZsq{}}\PY{p}{,}\PY{l+s+s1}{\PYZsq{}}\PY{l+s+s1}{😉}\PY{l+s+s1}{\PYZsq{}}\PY{p}{,}\PY{l+s+s1}{\PYZsq{}}\PY{l+s+s1}{👍😃✌}\PY{l+s+s1}{\PYZsq{}}\PY{p}{]}
\PY{n}{lista} \PY{o}{+} \PY{n}{lista2}  \PY{c+c1}{\PYZsh{} o que será que vai sair disso?? (que operação é essa?)}
\end{Verbatim}
\end{tcolorbox}

            \begin{tcolorbox}[breakable, size=fbox, boxrule=.5pt, pad at break*=1mm, opacityfill=0]
\prompt{Out}{outcolor}{15}{\boxspacing}
\begin{Verbatim}[commandchars=\\\{\}]
[1, 4, 9, 16, 25, 'a', 'b', 'c', 'd', 'texto também pode!', '😉', '👍😃✌']
\end{Verbatim}
\end{tcolorbox}
        
    Para mais informações sobre como usar listas, acesse este
\href{https://docs.python.org/3.7/tutorial/introduction.html\#lists}{link
para o tutorial do Python}.

    

    \hypertarget{extendendo-a-funcionalidade-com-bibliotecas}{%
\subsection{Extendendo a funcionalidade com
bibliotecas}\label{extendendo-a-funcionalidade-com-bibliotecas}}

A linguagem Python tem um número gigantesco de \textbf{bibliotecas}
(extensões da linguagem) especializadas em tarefas específicas. Elas
aumentam muito o funcionalidade e os recursos da liguagem. Além da
\href{https://docs.python.org/pt-br/3.7/library/index.html\#library-index}{Biblioteca
Padrão}, existe um grande número de bibliotecas produzidas pela
comunidade de código aberto, especialmente na
\href{https://scipy.org}{área científica}.

A filosofia da liguagem é diferene de várias outras liguagens, mantendo
apenas um \emph{core} (núcleo) mínimo de palavras reservadas
(``keywords'' e
\href{https://docs.python.org/pt-br/3.7/reference/lexical_analysis.html\#identifiers}{identificadores}),
deixando funcionalidades mais específicas para essas bibliotecas
especializadas. Assim, especialmente para o uso científico do Python,
nós precisaremo usar essas extensões.

Usaremos várias delas ao longo deste curso, principalmente
\href{https://numpy.org/doc/stable/user/index.html}{Numpy},
\href{https://scipy.org/}{Scipy},
\href{https://matplotlib.org/}{Matplotlib} e
\href{https://www.sympy.org/en/index.html}{SymPy}, além, claro do
próprio Jupyter, que também é uma extensão da linguagem, como resultado
da evolução do \href{https://ipython.org/}{IPython}.

Abaixo eu mostro como carregar algumas delas.

    \begin{tcolorbox}[breakable, size=fbox, boxrule=1pt, pad at break*=1mm,colback=cellbackground, colframe=cellborder]
\prompt{In}{incolor}{16}{\boxspacing}
\begin{Verbatim}[commandchars=\\\{\}]
\PY{k+kn}{from} \PY{n+nn}{numpy} \PY{k+kn}{import} \PY{n}{array}\PY{p}{,} \PY{n}{dot}\PY{p}{,} \PY{n}{outer}\PY{p}{,} \PY{n}{sqrt}\PY{p}{,} \PY{n}{matrix}\PY{p}{,} \PY{n}{linspace}\PY{p}{,} \PY{n}{sin}\PY{p}{,} \PY{n}{cos}
\PY{k+kn}{from} \PY{n+nn}{numpy}\PY{n+nn}{.}\PY{n+nn}{linalg} \PY{k+kn}{import} \PY{n}{eig}\PY{p}{,} \PY{n}{eigvals}
\PY{k+kn}{from} \PY{n+nn}{matplotlib}\PY{n+nn}{.}\PY{n+nn}{pyplot} \PY{k+kn}{import} \PY{n}{hist}\PY{p}{,} \PY{n}{plot}
\PY{o}{\PYZpc{}}\PY{k}{matplotlib} inline
\end{Verbatim}
\end{tcolorbox}

    Outra forma de fazer isso é:

    \begin{tcolorbox}[breakable, size=fbox, boxrule=1pt, pad at break*=1mm,colback=cellbackground, colframe=cellborder]
\prompt{In}{incolor}{17}{\boxspacing}
\begin{Verbatim}[commandchars=\\\{\}]
\PY{k+kn}{import} \PY{n+nn}{numpy} \PY{k}{as} \PY{n+nn}{np}        \PY{c+c1}{\PYZsh{} aqui atribui\PYZhy{}se um \PYZdq{}apelido\PYZdq{} (\PYZsq{}alias\PYZsq{}) para o nome da biblioteca}
\PY{k+kn}{import} \PY{n+nn}{numpy}\PY{n+nn}{.}\PY{n+nn}{linalg} \PY{k}{as} \PY{n+nn}{la}
\end{Verbatim}
\end{tcolorbox}

    \hypertarget{vetores}{%
\subsubsection{Vetores}\label{vetores}}

Praticamente toda linguagem de programação possui algum tipo de
estrutura para armazenar dados de um determinado tipo (geralmente
números, mas também caracteres). Geralmente são chamados de
``\textbf{arrays}'' (que no caso do Python, estão na biblioteca
\href{https://numpy.org/doc/stable/}{Numpy}, que foi carregado acima).
No nosso idioma eles são normalmente traduzido como ``vetores''. Esses
vetores são parecidos, mas \emph{nem sempre equivalentes} , ao conceito
de vetores usado na Física ou na Algebra Linear.

Podemos, porém, definir e usar formalmente algo análogo aos vetores que
temos usado neste curso, mesmo no sentido de espaços vetorias
complexos\ldots{}

Abaixo eu mostro como definir vetores \textbf{linhas} e
\textbf{colunas}, como os usados na MQ.

    \begin{tcolorbox}[breakable, size=fbox, boxrule=1pt, pad at break*=1mm,colback=cellbackground, colframe=cellborder]
\prompt{In}{incolor}{18}{\boxspacing}
\begin{Verbatim}[commandchars=\\\{\}]
\PY{n}{vl} \PY{o}{=} \PY{n}{array}\PY{p}{(}\PY{p}{[}\PY{l+m+mi}{1}\PY{p}{,}\PY{l+m+mi}{2}\PY{p}{,}\PY{l+m+mi}{3}\PY{p}{]}\PY{p}{)}  \PY{c+c1}{\PYZsh{} um vetor linha}
\PY{n}{vl}
\end{Verbatim}
\end{tcolorbox}

            \begin{tcolorbox}[breakable, size=fbox, boxrule=.5pt, pad at break*=1mm, opacityfill=0]
\prompt{Out}{outcolor}{18}{\boxspacing}
\begin{Verbatim}[commandchars=\\\{\}]
array([1, 2, 3])
\end{Verbatim}
\end{tcolorbox}
        
    \begin{tcolorbox}[breakable, size=fbox, boxrule=1pt, pad at break*=1mm,colback=cellbackground, colframe=cellborder]
\prompt{In}{incolor}{19}{\boxspacing}
\begin{Verbatim}[commandchars=\\\{\}]
\PY{n}{vc} \PY{o}{=} \PY{n}{array}\PY{p}{(}\PY{p}{[}\PY{p}{[}\PY{l+m+mi}{4}\PY{p}{]}\PY{p}{,}\PY{p}{[}\PY{l+m+mi}{5}\PY{p}{]}\PY{p}{,}\PY{p}{[}\PY{l+m+mi}{6}\PY{p}{]}\PY{p}{]}\PY{p}{)}  \PY{c+c1}{\PYZsh{} um vetor coluna}
\PY{n}{vc}
\end{Verbatim}
\end{tcolorbox}

            \begin{tcolorbox}[breakable, size=fbox, boxrule=.5pt, pad at break*=1mm, opacityfill=0]
\prompt{Out}{outcolor}{19}{\boxspacing}
\begin{Verbatim}[commandchars=\\\{\}]
array([[4],
       [5],
       [6]])
\end{Verbatim}
\end{tcolorbox}
        
    \begin{tcolorbox}[breakable, size=fbox, boxrule=1pt, pad at break*=1mm,colback=cellbackground, colframe=cellborder]
\prompt{In}{incolor}{20}{\boxspacing}
\begin{Verbatim}[commandchars=\\\{\}]
\PY{n}{dot}\PY{p}{(}\PY{n}{vl}\PY{p}{,}\PY{n}{vc}\PY{p}{)} \PY{c+c1}{\PYZsh{} produto escalar (produto interno): retorna um escalar}
\end{Verbatim}
\end{tcolorbox}

            \begin{tcolorbox}[breakable, size=fbox, boxrule=.5pt, pad at break*=1mm, opacityfill=0]
\prompt{Out}{outcolor}{20}{\boxspacing}
\begin{Verbatim}[commandchars=\\\{\}]
array([32])
\end{Verbatim}
\end{tcolorbox}
        
    \begin{tcolorbox}[breakable, size=fbox, boxrule=1pt, pad at break*=1mm,colback=cellbackground, colframe=cellborder]
\prompt{In}{incolor}{21}{\boxspacing}
\begin{Verbatim}[commandchars=\\\{\}]
\PY{n}{dot}\PY{p}{(}\PY{n}{vc}\PY{p}{,}\PY{n}{vl}\PY{p}{)}  \PY{c+c1}{\PYZsh{} CUIDADO!! A ordem faz diferença... (porque??!)}
\end{Verbatim}
\end{tcolorbox}

    \begin{Verbatim}[commandchars=\\\{\}]

        ---------------------------------------------------------------------------

        ValueError                                Traceback (most recent call last)

        <ipython-input-21-84108f792119> in <module>
    ----> 1 dot(vc,vl)  \# CUIDADO!! A ordem faz diferença{\ldots} (porque??!)
    

        <\_\_array\_function\_\_ internals> in dot(*args, **kwargs)
    

        ValueError: shapes (3,1) and (3,) not aligned: 1 (dim 1) != 3 (dim 0)

    \end{Verbatim}

    \begin{tcolorbox}[breakable, size=fbox, boxrule=1pt, pad at break*=1mm,colback=cellbackground, colframe=cellborder]
\prompt{In}{incolor}{22}{\boxspacing}
\begin{Verbatim}[commandchars=\\\{\}]
\PY{n}{outer} \PY{p}{(}\PY{n}{vc}\PY{p}{,}\PY{n}{vl}\PY{p}{)}  \PY{c+c1}{\PYZsh{} retorna um operador linear (não um escalar!)}
\end{Verbatim}
\end{tcolorbox}

            \begin{tcolorbox}[breakable, size=fbox, boxrule=.5pt, pad at break*=1mm, opacityfill=0]
\prompt{Out}{outcolor}{22}{\boxspacing}
\begin{Verbatim}[commandchars=\\\{\}]
array([[ 4,  8, 12],
       [ 5, 10, 15],
       [ 6, 12, 18]])
\end{Verbatim}
\end{tcolorbox}
        
    \begin{tcolorbox}[breakable, size=fbox, boxrule=1pt, pad at break*=1mm,colback=cellbackground, colframe=cellborder]
\prompt{In}{incolor}{23}{\boxspacing}
\begin{Verbatim}[commandchars=\\\{\}]
\PY{n}{outer} \PY{p}{(}\PY{n}{vl}\PY{p}{,}\PY{n}{vc}\PY{p}{)} \PY{c+c1}{\PYZsh{} note que a ordem dos vetores mudou, assim como a matriz do operador (como?)}
\end{Verbatim}
\end{tcolorbox}

            \begin{tcolorbox}[breakable, size=fbox, boxrule=.5pt, pad at break*=1mm, opacityfill=0]
\prompt{Out}{outcolor}{23}{\boxspacing}
\begin{Verbatim}[commandchars=\\\{\}]
array([[ 4,  5,  6],
       [ 8, 10, 12],
       [12, 15, 18]])
\end{Verbatim}
\end{tcolorbox}
        
    \hypertarget{vetores-de-nuxfameros-complexos-como-no-espauxe7o-de-hilbert}{%
\subsubsection{Vetores de números complexos (como no espaço de
Hilbert)}\label{vetores-de-nuxfameros-complexos-como-no-espauxe7o-de-hilbert}}

    \begin{tcolorbox}[breakable, size=fbox, boxrule=1pt, pad at break*=1mm,colback=cellbackground, colframe=cellborder]
\prompt{In}{incolor}{24}{\boxspacing}
\begin{Verbatim}[commandchars=\\\{\}]
\PY{n}{v1} \PY{o}{=} \PY{n}{array}\PY{p}{(}\PY{p}{[}\PY{l+m+mi}{1}\PY{o}{+}\PY{l+m+mi}{2}\PY{n}{j}\PY{p}{,} \PY{l+m+mi}{3}\PY{o}{+}\PY{l+m+mi}{2}\PY{n}{j}\PY{p}{,} \PY{l+m+mi}{5}\PY{o}{+}\PY{l+m+mi}{1}\PY{n}{j}\PY{p}{,} \PY{l+m+mi}{4}\PY{o}{+}\PY{l+m+mi}{0}\PY{n}{j}\PY{p}{]}\PY{p}{)}
\end{Verbatim}
\end{tcolorbox}

    \begin{tcolorbox}[breakable, size=fbox, boxrule=1pt, pad at break*=1mm,colback=cellbackground, colframe=cellborder]
\prompt{In}{incolor}{25}{\boxspacing}
\begin{Verbatim}[commandchars=\\\{\}]
\PY{n}{v1}\PY{o}{*}\PY{n}{v1}  \PY{c+c1}{\PYZsh{} produto (direto) de dois vetores complexos (você entendeu o resultado??!)}
\end{Verbatim}
\end{tcolorbox}

            \begin{tcolorbox}[breakable, size=fbox, boxrule=.5pt, pad at break*=1mm, opacityfill=0]
\prompt{Out}{outcolor}{25}{\boxspacing}
\begin{Verbatim}[commandchars=\\\{\}]
array([-3. +4.j,  5.+12.j, 24.+10.j, 16. +0.j])
\end{Verbatim}
\end{tcolorbox}
        
    \begin{tcolorbox}[breakable, size=fbox, boxrule=1pt, pad at break*=1mm,colback=cellbackground, colframe=cellborder]
\prompt{In}{incolor}{26}{\boxspacing}
\begin{Verbatim}[commandchars=\\\{\}]
\PY{n}{v1}\PY{o}{*}\PY{n}{v1}\PY{o}{.}\PY{n}{conjugate}\PY{p}{(}\PY{p}{)}  \PY{c+c1}{\PYZsh{} multiplicando pelo complexo conjutado (retorna um real)}
\end{Verbatim}
\end{tcolorbox}

            \begin{tcolorbox}[breakable, size=fbox, boxrule=.5pt, pad at break*=1mm, opacityfill=0]
\prompt{Out}{outcolor}{26}{\boxspacing}
\begin{Verbatim}[commandchars=\\\{\}]
array([ 5.+0.j, 13.+0.j, 26.+0.j, 16.+0.j])
\end{Verbatim}
\end{tcolorbox}
        
    \begin{tcolorbox}[breakable, size=fbox, boxrule=1pt, pad at break*=1mm,colback=cellbackground, colframe=cellborder]
\prompt{In}{incolor}{27}{\boxspacing}
\begin{Verbatim}[commandchars=\\\{\}]
\PY{p}{(}\PY{l+m+mi}{1}\PY{o}{+}\PY{l+m+mi}{2}\PY{n}{j}\PY{p}{)}\PY{o}{*}\PY{p}{(}\PY{l+m+mi}{1}\PY{o}{+}\PY{l+m+mi}{2}\PY{n}{j}\PY{p}{)}
\end{Verbatim}
\end{tcolorbox}

            \begin{tcolorbox}[breakable, size=fbox, boxrule=.5pt, pad at break*=1mm, opacityfill=0]
\prompt{Out}{outcolor}{27}{\boxspacing}
\begin{Verbatim}[commandchars=\\\{\}]
(-3+4j)
\end{Verbatim}
\end{tcolorbox}
        
    \begin{tcolorbox}[breakable, size=fbox, boxrule=1pt, pad at break*=1mm,colback=cellbackground, colframe=cellborder]
\prompt{In}{incolor}{28}{\boxspacing}
\begin{Verbatim}[commandchars=\\\{\}]
\PY{n}{dot}\PY{p}{(}\PY{n}{v1}\PY{o}{.}\PY{n}{conjugate}\PY{p}{(}\PY{p}{)}\PY{p}{,}\PY{n}{v1}\PY{p}{)}   \PY{c+c1}{\PYZsh{} use o dot() para obter o produtor interno (escalar)}
\end{Verbatim}
\end{tcolorbox}

            \begin{tcolorbox}[breakable, size=fbox, boxrule=.5pt, pad at break*=1mm, opacityfill=0]
\prompt{Out}{outcolor}{28}{\boxspacing}
\begin{Verbatim}[commandchars=\\\{\}]
(60+0j)
\end{Verbatim}
\end{tcolorbox}
        
    \hypertarget{matrizes}{%
\subsubsection{Matrizes}\label{matrizes}}

Podemos também, claro, definir matrizes, que são objetos muito
importantes neste curso. Na linguagem Python (usando a Numpy), podemos
definir matrizes de duas formas. Uma é usando o mesmo comando
\textbf{array}, que, na verdade, funciona para ``vetores
multdimensionais'' dentro da linguagem. A outra forma será mostrada
também nos exemplos.

    \begin{tcolorbox}[breakable, size=fbox, boxrule=1pt, pad at break*=1mm,colback=cellbackground, colframe=cellborder]
\prompt{In}{incolor}{29}{\boxspacing}
\begin{Verbatim}[commandchars=\\\{\}]
\PY{c+c1}{\PYZsh{} a two\PYZhy{}dimensional array}
\PY{n}{m1} \PY{o}{=} \PY{n}{array}\PY{p}{(}\PY{p}{[}\PY{p}{[}\PY{l+m+mi}{1}\PY{p}{,}\PY{l+m+mi}{0}\PY{p}{]}\PY{p}{,}\PY{p}{[}\PY{l+m+mi}{1}\PY{p}{,}\PY{l+m+mi}{2}\PY{p}{]}\PY{p}{]}\PY{p}{)}
\PY{n}{m1}
\end{Verbatim}
\end{tcolorbox}

            \begin{tcolorbox}[breakable, size=fbox, boxrule=.5pt, pad at break*=1mm, opacityfill=0]
\prompt{Out}{outcolor}{29}{\boxspacing}
\begin{Verbatim}[commandchars=\\\{\}]
array([[1, 0],
       [1, 2]])
\end{Verbatim}
\end{tcolorbox}
        
    \begin{tcolorbox}[breakable, size=fbox, boxrule=1pt, pad at break*=1mm,colback=cellbackground, colframe=cellborder]
\prompt{In}{incolor}{30}{\boxspacing}
\begin{Verbatim}[commandchars=\\\{\}]
\PY{c+c1}{\PYZsh{} pode calcular a matriz transpostas, com o método T}
\PY{n}{m1}\PY{o}{.}\PY{n}{T}
\end{Verbatim}
\end{tcolorbox}

            \begin{tcolorbox}[breakable, size=fbox, boxrule=.5pt, pad at break*=1mm, opacityfill=0]
\prompt{Out}{outcolor}{30}{\boxspacing}
\begin{Verbatim}[commandchars=\\\{\}]
array([[1, 1],
       [0, 2]])
\end{Verbatim}
\end{tcolorbox}
        
    \begin{tcolorbox}[breakable, size=fbox, boxrule=1pt, pad at break*=1mm,colback=cellbackground, colframe=cellborder]
\prompt{In}{incolor}{31}{\boxspacing}
\begin{Verbatim}[commandchars=\\\{\}]
\PY{c+c1}{\PYZsh{} podemos também calcular (facilmente!) os autovalores e autovetores!! }
\PY{n}{eig}\PY{p}{(}\PY{n}{m1}\PY{p}{)}
\end{Verbatim}
\end{tcolorbox}

            \begin{tcolorbox}[breakable, size=fbox, boxrule=.5pt, pad at break*=1mm, opacityfill=0]
\prompt{Out}{outcolor}{31}{\boxspacing}
\begin{Verbatim}[commandchars=\\\{\}]
(array([2., 1.]),
 array([[ 0.        ,  0.70710678],
        [ 1.        , -0.70710678]]))
\end{Verbatim}
\end{tcolorbox}
        
    \begin{tcolorbox}[breakable, size=fbox, boxrule=1pt, pad at break*=1mm,colback=cellbackground, colframe=cellborder]
\prompt{In}{incolor}{32}{\boxspacing}
\begin{Verbatim}[commandchars=\\\{\}]
\PY{c+c1}{\PYZsh{} é fácil verificar o resultado também}
\PY{n}{dot}\PY{p}{(}\PY{n}{m1}\PY{p}{,}\PY{n}{array}\PY{p}{(}\PY{p}{[} \PY{l+m+mi}{0}\PY{p}{,} \PY{l+m+mi}{1}\PY{p}{]}\PY{p}{)}\PY{p}{)}
\end{Verbatim}
\end{tcolorbox}

            \begin{tcolorbox}[breakable, size=fbox, boxrule=.5pt, pad at break*=1mm, opacityfill=0]
\prompt{Out}{outcolor}{32}{\boxspacing}
\begin{Verbatim}[commandchars=\\\{\}]
array([0, 2])
\end{Verbatim}
\end{tcolorbox}
        
    \hypertarget{buscando-ajuda-com-os-comandos-e-help}{%
\subsubsection{\texorpdfstring{Buscando ajuda com os comandos
``\emph{?}'' e " \emph{help()}
"}{Buscando ajuda com os comandos ``?'' e " help() "}}\label{buscando-ajuda-com-os-comandos-e-help}}

    \begin{tcolorbox}[breakable, size=fbox, boxrule=1pt, pad at break*=1mm,colback=cellbackground, colframe=cellborder]
\prompt{In}{incolor}{33}{\boxspacing}
\begin{Verbatim}[commandchars=\\\{\}]
\PY{c+c1}{\PYZsh{} use o sinal de interrogação para acessar o comando help() }
eig\PY{o}{?}
\end{Verbatim}
\end{tcolorbox}

    \begin{tcolorbox}[breakable, size=fbox, boxrule=1pt, pad at break*=1mm,colback=cellbackground, colframe=cellborder]
\prompt{In}{incolor}{34}{\boxspacing}
\begin{Verbatim}[commandchars=\\\{\}]
\PY{c+c1}{\PYZsh{} assim também funciona... (mas o resultado é diferente)}
\PY{n}{help}\PY{p}{(}\PY{n}{eig}\PY{p}{)}
\end{Verbatim}
\end{tcolorbox}

    \begin{Verbatim}[commandchars=\\\{\}]
Help on function eig in module numpy.linalg:

eig(a)
    Compute the eigenvalues and right eigenvectors of a square array.

    Parameters
    ----------
    a : ({\ldots}, M, M) array
        Matrices for which the eigenvalues and right eigenvectors will
        be computed

    Returns
    -------
    w : ({\ldots}, M) array
        The eigenvalues, each repeated according to its multiplicity.
        The eigenvalues are not necessarily ordered. The resulting
        array will be of complex type, unless the imaginary part is
        zero in which case it will be cast to a real type. When `a`
        is real the resulting eigenvalues will be real (0 imaginary
        part) or occur in conjugate pairs

    v : ({\ldots}, M, M) array
        The normalized (unit "length") eigenvectors, such that the
        column ``v[:,i]`` is the eigenvector corresponding to the
        eigenvalue ``w[i]``.

    Raises
    ------
    LinAlgError
        If the eigenvalue computation does not converge.

    See Also
    --------
    eigvals : eigenvalues of a non-symmetric array.

    eigh : eigenvalues and eigenvectors of a real symmetric or complex
           Hermitian (conjugate symmetric) array.

    eigvalsh : eigenvalues of a real symmetric or complex Hermitian
               (conjugate symmetric) array.

    Notes
    -----

    .. versionadded:: 1.8.0

    Broadcasting rules apply, see the `numpy.linalg` documentation for
    details.

    This is implemented using the ``\_geev`` LAPACK routines which compute
    the eigenvalues and eigenvectors of general square arrays.

    The number `w` is an eigenvalue of `a` if there exists a vector
    `v` such that ``dot(a,v) = w * v``. Thus, the arrays `a`, `w`, and
    `v` satisfy the equations ``dot(a[:,:], v[:,i]) = w[i] * v[:,i]``
    for :math:`i \textbackslash{}in \textbackslash{}\{0,{\ldots},M-1\textbackslash{}\}`.

    The array `v` of eigenvectors may not be of maximum rank, that is, some
    of the columns may be linearly dependent, although round-off error may
    obscure that fact. If the eigenvalues are all different, then theoretically
    the eigenvectors are linearly independent. Likewise, the (complex-valued)
    matrix of eigenvectors `v` is unitary if the matrix `a` is normal, i.e.,
    if ``dot(a, a.H) = dot(a.H, a)``, where `a.H` denotes the conjugate
    transpose of `a`.

    Finally, it is emphasized that `v` consists of the *right* (as in
    right-hand side) eigenvectors of `a`.  A vector `y` satisfying
    ``dot(y.T, a) = z * y.T`` for some number `z` is called a *left*
    eigenvector of `a`, and, in general, the left and right eigenvectors
    of a matrix are not necessarily the (perhaps conjugate) transposes
    of each other.

    References
    ----------
    G. Strang, *Linear Algebra and Its Applications*, 2nd Ed., Orlando, FL,
    Academic Press, Inc., 1980, Various pp.

    Examples
    --------
    >>> from numpy import linalg as LA

    (Almost) trivial example with real e-values and e-vectors.

    >>> w, v = LA.eig(np.diag((1, 2, 3)))
    >>> w; v
    array([1., 2., 3.])
    array([[1., 0., 0.],
           [0., 1., 0.],
           [0., 0., 1.]])

    Real matrix possessing complex e-values and e-vectors; note that the
    e-values are complex conjugates of each other.

    >>> w, v = LA.eig(np.array([[1, -1], [1, 1]]))
    >>> w; v
    array([1.+1.j, 1.-1.j])
    array([[0.70710678+0.j        , 0.70710678-0.j        ],
           [0.        -0.70710678j, 0.        +0.70710678j]])

    Complex-valued matrix with real e-values (but complex-valued e-vectors);
    note that ``a.conj().T == a``, i.e., `a` is Hermitian.

    >>> a = np.array([[1, 1j], [-1j, 1]])
    >>> w, v = LA.eig(a)
    >>> w; v
    array([2.+0.j, 0.+0.j])
    array([[ 0.        +0.70710678j,  0.70710678+0.j        ], \# may vary
           [ 0.70710678+0.j        , -0.        +0.70710678j]])

    Be careful about round-off error!

    >>> a = np.array([[1 + 1e-9, 0], [0, 1 - 1e-9]])
    >>> \# Theor. e-values are 1 +/- 1e-9
    >>> w, v = LA.eig(a)
    >>> w; v
    array([1., 1.])
    array([[1., 0.],
           [0., 1.]])

    \end{Verbatim}

    

    \hypertarget{exemplos-e-exercuxedcios-para-praticar}{%
\section{Exemplos e exercícios para
praticar}\label{exemplos-e-exercuxedcios-para-praticar}}

    \hypertarget{exemplo-1}{%
\subsection{Exemplo 1}\label{exemplo-1}}

Lembra aquele exemplo\ldots{} \[
M = \begin{pmatrix} 1 & 2 \\ 1 & 0 \end{pmatrix}
\]

    \begin{tcolorbox}[breakable, size=fbox, boxrule=1pt, pad at break*=1mm,colback=cellbackground, colframe=cellborder]
\prompt{In}{incolor}{35}{\boxspacing}
\begin{Verbatim}[commandchars=\\\{\}]
\PY{n}{M} \PY{o}{=} \PY{n}{matrix}\PY{p}{(}\PY{p}{[}\PY{p}{[}\PY{l+m+mi}{1}\PY{p}{,}\PY{l+m+mi}{2}\PY{p}{]}\PY{p}{,}\PY{p}{[}\PY{l+m+mi}{1}\PY{p}{,}\PY{l+m+mi}{0}\PY{p}{]}\PY{p}{]}\PY{p}{)}
\PY{n}{M}
\end{Verbatim}
\end{tcolorbox}

            \begin{tcolorbox}[breakable, size=fbox, boxrule=.5pt, pad at break*=1mm, opacityfill=0]
\prompt{Out}{outcolor}{35}{\boxspacing}
\begin{Verbatim}[commandchars=\\\{\}]
matrix([[1, 2],
        [1, 0]])
\end{Verbatim}
\end{tcolorbox}
        
    Quais os autovalores e autovetores?

    \begin{tcolorbox}[breakable, size=fbox, boxrule=1pt, pad at break*=1mm,colback=cellbackground, colframe=cellborder]
\prompt{In}{incolor}{36}{\boxspacing}
\begin{Verbatim}[commandchars=\\\{\}]
\PY{n}{eig}\PY{p}{(}\PY{n}{M}\PY{p}{)}
\end{Verbatim}
\end{tcolorbox}

            \begin{tcolorbox}[breakable, size=fbox, boxrule=.5pt, pad at break*=1mm, opacityfill=0]
\prompt{Out}{outcolor}{36}{\boxspacing}
\begin{Verbatim}[commandchars=\\\{\}]
(array([ 2., -1.]),
 matrix([[ 0.89442719, -0.70710678],
         [ 0.4472136 ,  0.70710678]]))
\end{Verbatim}
\end{tcolorbox}
        
    Vamos explorar os elementos desse objeto, atribuindo-o a uma variável.

    \begin{tcolorbox}[breakable, size=fbox, boxrule=1pt, pad at break*=1mm,colback=cellbackground, colframe=cellborder]
\prompt{In}{incolor}{37}{\boxspacing}
\begin{Verbatim}[commandchars=\\\{\}]
\PY{n}{autos} \PY{o}{=} \PY{n}{eig}\PY{p}{(}\PY{n}{M}\PY{p}{)}
\end{Verbatim}
\end{tcolorbox}

    Ao fazer isso, a função eig(M) retorna um objeto diferente dos que vimos
até agora, chamado \textbf{Tuple}. Esse objeto é também uma coleção de
elementos ordenados, parecido com uma lista, mas é indicado pelo símbolo
de parênteses, ao invés dos colchetes. Podemos acessar os elementos de
um \textbf{tuple}, de forma parecida com as \textbf{listas} e
\textbf{arrays}. Veja os exemplos:

    \begin{tcolorbox}[breakable, size=fbox, boxrule=1pt, pad at break*=1mm,colback=cellbackground, colframe=cellborder]
\prompt{In}{incolor}{38}{\boxspacing}
\begin{Verbatim}[commandchars=\\\{\}]
\PY{n}{autos}\PY{p}{[}\PY{l+m+mi}{0}\PY{p}{]}
\end{Verbatim}
\end{tcolorbox}

            \begin{tcolorbox}[breakable, size=fbox, boxrule=.5pt, pad at break*=1mm, opacityfill=0]
\prompt{Out}{outcolor}{38}{\boxspacing}
\begin{Verbatim}[commandchars=\\\{\}]
array([ 2., -1.])
\end{Verbatim}
\end{tcolorbox}
        
    \begin{tcolorbox}[breakable, size=fbox, boxrule=1pt, pad at break*=1mm,colback=cellbackground, colframe=cellborder]
\prompt{In}{incolor}{39}{\boxspacing}
\begin{Verbatim}[commandchars=\\\{\}]
\PY{n}{autos}\PY{p}{[}\PY{l+m+mi}{1}\PY{p}{]}
\end{Verbatim}
\end{tcolorbox}

            \begin{tcolorbox}[breakable, size=fbox, boxrule=.5pt, pad at break*=1mm, opacityfill=0]
\prompt{Out}{outcolor}{39}{\boxspacing}
\begin{Verbatim}[commandchars=\\\{\}]
matrix([[ 0.89442719, -0.70710678],
        [ 0.4472136 ,  0.70710678]])
\end{Verbatim}
\end{tcolorbox}
        
    Observer que este elemento (o segundo elemento do tuple) é uma matriz
Numpy. Seus elementos podem ser acessados como qualquer matriz (ou
array) no numpy. Por exemplo, para acessar o primeiro vetor linha

    \begin{tcolorbox}[breakable, size=fbox, boxrule=1pt, pad at break*=1mm,colback=cellbackground, colframe=cellborder]
\prompt{In}{incolor}{40}{\boxspacing}
\begin{Verbatim}[commandchars=\\\{\}]
\PY{n}{autos}\PY{p}{[}\PY{l+m+mi}{1}\PY{p}{]}\PY{p}{[}\PY{l+m+mi}{0}\PY{p}{]}
\end{Verbatim}
\end{tcolorbox}

            \begin{tcolorbox}[breakable, size=fbox, boxrule=.5pt, pad at break*=1mm, opacityfill=0]
\prompt{Out}{outcolor}{40}{\boxspacing}
\begin{Verbatim}[commandchars=\\\{\}]
matrix([[ 0.89442719, -0.70710678]])
\end{Verbatim}
\end{tcolorbox}
        
    \begin{tcolorbox}[breakable, size=fbox, boxrule=1pt, pad at break*=1mm,colback=cellbackground, colframe=cellborder]
\prompt{In}{incolor}{41}{\boxspacing}
\begin{Verbatim}[commandchars=\\\{\}]
\PY{n}{dot}\PY{p}{(}\PY{n}{autos}\PY{p}{[}\PY{l+m+mi}{1}\PY{p}{]}\PY{p}{[}\PY{l+m+mi}{0}\PY{p}{]}\PY{p}{,}\PY{n}{M}\PY{p}{)}  \PY{c+c1}{\PYZsh{} multiplicação à esquerda de um vetor linha pelo operador M}
\end{Verbatim}
\end{tcolorbox}

            \begin{tcolorbox}[breakable, size=fbox, boxrule=.5pt, pad at break*=1mm, opacityfill=0]
\prompt{Out}{outcolor}{41}{\boxspacing}
\begin{Verbatim}[commandchars=\\\{\}]
matrix([[0.18732041, 1.78885438]])
\end{Verbatim}
\end{tcolorbox}
        
    Você já deve ter notado que \textbf{eig(M)} retorna os autovalores de
\textbf{M}, no primeiro elemento do tuple, e os autovetores, no segundo
elemento do tuple. Os autovetores estão na forma de vetores colunas.

    \begin{tcolorbox}[breakable, size=fbox, boxrule=1pt, pad at break*=1mm,colback=cellbackground, colframe=cellborder]
\prompt{In}{incolor}{42}{\boxspacing}
\begin{Verbatim}[commandchars=\\\{\}]
\PY{n}{autoV} \PY{o}{=} \PY{n}{autos}\PY{p}{[}\PY{l+m+mi}{1}\PY{p}{]}   \PY{c+c1}{\PYZsh{} transpondo a matriz}
\PY{n}{autoV}
\end{Verbatim}
\end{tcolorbox}

            \begin{tcolorbox}[breakable, size=fbox, boxrule=.5pt, pad at break*=1mm, opacityfill=0]
\prompt{Out}{outcolor}{42}{\boxspacing}
\begin{Verbatim}[commandchars=\\\{\}]
matrix([[ 0.89442719, -0.70710678],
        [ 0.4472136 ,  0.70710678]])
\end{Verbatim}
\end{tcolorbox}
        
    podemos calcular a norma dos vetores colunas através do método
\textbf{numpy.linalg.norm()}, que aqui pode ser acessado de uma forma
abrevidas, pois já carregamos essa numpy.linalg com o alias
``\(\textrm{la}\)''

    \begin{tcolorbox}[breakable, size=fbox, boxrule=1pt, pad at break*=1mm,colback=cellbackground, colframe=cellborder]
\prompt{In}{incolor}{43}{\boxspacing}
\begin{Verbatim}[commandchars=\\\{\}]
\PY{n}{la}\PY{o}{.}\PY{n}{norm}\PY{p}{(}\PY{n}{autoV}\PY{o}{.}\PY{n}{T}\PY{p}{[}\PY{l+m+mi}{1}\PY{p}{]}\PY{p}{)}
\end{Verbatim}
\end{tcolorbox}

            \begin{tcolorbox}[breakable, size=fbox, boxrule=.5pt, pad at break*=1mm, opacityfill=0]
\prompt{Out}{outcolor}{43}{\boxspacing}
\begin{Verbatim}[commandchars=\\\{\}]
1.0
\end{Verbatim}
\end{tcolorbox}
        
    Talvez lhe seja mais familiar, se divindo por \(\sqrt2\) e \(\sqrt5\)
(fatorando esses termos), como fizemos nas aulas.

    \begin{tcolorbox}[breakable, size=fbox, boxrule=1pt, pad at break*=1mm,colback=cellbackground, colframe=cellborder]
\prompt{In}{incolor}{44}{\boxspacing}
\begin{Verbatim}[commandchars=\\\{\}]
\PY{n}{autoV}\PY{o}{.}\PY{n}{T}\PY{p}{[}\PY{l+m+mi}{1}\PY{p}{]}\PY{o}{.}\PY{n}{T}\PY{o}{/}\PY{n}{sqrt}\PY{p}{(}\PY{l+m+mi}{2}\PY{p}{)}
\end{Verbatim}
\end{tcolorbox}

            \begin{tcolorbox}[breakable, size=fbox, boxrule=.5pt, pad at break*=1mm, opacityfill=0]
\prompt{Out}{outcolor}{44}{\boxspacing}
\begin{Verbatim}[commandchars=\\\{\}]
matrix([[-0.5],
        [ 0.5]])
\end{Verbatim}
\end{tcolorbox}
        
    \begin{tcolorbox}[breakable, size=fbox, boxrule=1pt, pad at break*=1mm,colback=cellbackground, colframe=cellborder]
\prompt{In}{incolor}{45}{\boxspacing}
\begin{Verbatim}[commandchars=\\\{\}]
\PY{n}{autoV}\PY{o}{.}\PY{n}{T}\PY{p}{[}\PY{l+m+mi}{0}\PY{p}{]}\PY{o}{.}\PY{n}{T}\PY{o}{/}\PY{n}{sqrt}\PY{p}{(}\PY{l+m+mi}{5}\PY{p}{)}
\end{Verbatim}
\end{tcolorbox}

            \begin{tcolorbox}[breakable, size=fbox, boxrule=.5pt, pad at break*=1mm, opacityfill=0]
\prompt{Out}{outcolor}{45}{\boxspacing}
\begin{Verbatim}[commandchars=\\\{\}]
matrix([[0.4],
        [0.2]])
\end{Verbatim}
\end{tcolorbox}
        
    podemos agora testar se os autovetores estão corretos, aplicando o
operador sobre eles

    \begin{tcolorbox}[breakable, size=fbox, boxrule=1pt, pad at break*=1mm,colback=cellbackground, colframe=cellborder]
\prompt{In}{incolor}{46}{\boxspacing}
\begin{Verbatim}[commandchars=\\\{\}]
\PY{n}{dot}\PY{p}{(}\PY{n}{M}\PY{p}{,}\PY{n}{autoV}\PY{o}{.}\PY{n}{T}\PY{p}{[}\PY{l+m+mi}{1}\PY{p}{]}\PY{o}{.}\PY{n}{T}\PY{p}{)}\PY{o}{/}\PY{n}{sqrt}\PY{p}{(}\PY{l+m+mi}{2}\PY{p}{)}  \PY{c+c1}{\PYZsh{} resulta no autovetor multplicado por seu autovalor}
\end{Verbatim}
\end{tcolorbox}

            \begin{tcolorbox}[breakable, size=fbox, boxrule=.5pt, pad at break*=1mm, opacityfill=0]
\prompt{Out}{outcolor}{46}{\boxspacing}
\begin{Verbatim}[commandchars=\\\{\}]
matrix([[ 0.5],
        [-0.5]])
\end{Verbatim}
\end{tcolorbox}
        
    \begin{tcolorbox}[breakable, size=fbox, boxrule=1pt, pad at break*=1mm,colback=cellbackground, colframe=cellborder]
\prompt{In}{incolor}{47}{\boxspacing}
\begin{Verbatim}[commandchars=\\\{\}]
\PY{n}{dot}\PY{p}{(}\PY{n}{M}\PY{p}{,}\PY{n}{autoV}\PY{o}{.}\PY{n}{T}\PY{p}{[}\PY{l+m+mi}{0}\PY{p}{]}\PY{o}{.}\PY{n}{T}\PY{p}{)}\PY{o}{/}\PY{n}{sqrt}\PY{p}{(}\PY{l+m+mi}{5}\PY{p}{)}  \PY{c+c1}{\PYZsh{} resulta no autovetor multplicado por seu autovalor}
\end{Verbatim}
\end{tcolorbox}

            \begin{tcolorbox}[breakable, size=fbox, boxrule=.5pt, pad at break*=1mm, opacityfill=0]
\prompt{Out}{outcolor}{47}{\boxspacing}
\begin{Verbatim}[commandchars=\\\{\}]
matrix([[0.8],
        [0.4]])
\end{Verbatim}
\end{tcolorbox}
        
    Vale lembrar que um autovetor continua sendo um autovetor quando
multiplicado por um escalar (a direção não muda!), portanto fatores
multiplicativos não são relevanes para autovetores. Você pode observar,
por exemplo, que o o primeiro autovetores acima, tem um fator de 2 com
relação ao que havíamos determinado manualmente, na aula.

    \textbf{Transformação de similaridade}

Podemos faze mais um exemplo de uso desse rescusos para demostrar o
método de diagonalização discutido nas últimas aulas. Vimos que a a
transformação \textbf{S} pode ser construída com os autovetores do
operador (neste caso, a matriz \textbf{M}). Neste caso, a matriz
\textbf{S} será exatamente a matriz autoV, calculada acima\ldots{}

    \begin{tcolorbox}[breakable, size=fbox, boxrule=1pt, pad at break*=1mm,colback=cellbackground, colframe=cellborder]
\prompt{In}{incolor}{48}{\boxspacing}
\begin{Verbatim}[commandchars=\\\{\}]
\PY{n}{S} \PY{o}{=} \PY{n}{autoV}
\PY{n}{Si} \PY{o}{=} \PY{n}{la}\PY{o}{.}\PY{n}{inv}\PY{p}{(}\PY{n}{S}\PY{p}{)}  \PY{c+c1}{\PYZsh{} calcula a matriz inversa}

\PY{n}{dot}\PY{p}{(}\PY{n}{Si}\PY{p}{,}\PY{n}{dot}\PY{p}{(}\PY{n}{M}\PY{p}{,}\PY{n}{S}\PY{p}{)}\PY{p}{)}
\end{Verbatim}
\end{tcolorbox}

            \begin{tcolorbox}[breakable, size=fbox, boxrule=.5pt, pad at break*=1mm, opacityfill=0]
\prompt{Out}{outcolor}{48}{\boxspacing}
\begin{Verbatim}[commandchars=\\\{\}]
matrix([[ 2.00000000e+00,  2.22044605e-16],
        [ 0.00000000e+00, -1.00000000e+00]])
\end{Verbatim}
\end{tcolorbox}
        
    \begin{tcolorbox}[breakable, size=fbox, boxrule=1pt, pad at break*=1mm,colback=cellbackground, colframe=cellborder]
\prompt{In}{incolor}{49}{\boxspacing}
\begin{Verbatim}[commandchars=\\\{\}]
\PY{c+c1}{\PYZsh{} você pode \PYZdq{}limpar\PYZdq{} os erros numéricos de arredondamento, se preferir}
\PY{n}{dot}\PY{p}{(}\PY{n}{Si}\PY{p}{,}\PY{n}{dot}\PY{p}{(}\PY{n}{M}\PY{p}{,}\PY{n}{S}\PY{p}{)}\PY{p}{)}\PY{o}{.}\PY{n}{round}\PY{p}{(}\PY{p}{)}
\end{Verbatim}
\end{tcolorbox}

            \begin{tcolorbox}[breakable, size=fbox, boxrule=.5pt, pad at break*=1mm, opacityfill=0]
\prompt{Out}{outcolor}{49}{\boxspacing}
\begin{Verbatim}[commandchars=\\\{\}]
matrix([[ 2.,  0.],
        [ 0., -1.]])
\end{Verbatim}
\end{tcolorbox}
        
    Onde podemos ver que a matriz está agora na forma sua forma diagonal. 😉

    \hypertarget{exemplo-2}{%
\subsection{Exemplo 2}\label{exemplo-2}}

Para praticar um pouco, agora é sua vez! Vamos usar as matrizes do
\textbf{Problema 3} da \textbf{Lista 3}:

\[
A=
\begin{pmatrix}
-1 & 2i & 0 \\
 0 &  4 & 0 \\
 1 &  0 & 1
\end{pmatrix}
\]

\[
B=
\begin{pmatrix}
 0 & 2  & i \\
-i & 2i & 0 \\
 0 &  1 & 4
\end{pmatrix}
\]

    \textbf{Agora é sua vez:} comece escrevendo essas matriz num formato que
a linguagem Python entenda.

    \begin{tcolorbox}[breakable, size=fbox, boxrule=1pt, pad at break*=1mm,colback=cellbackground, colframe=cellborder]
\prompt{In}{incolor}{ }{\boxspacing}
\begin{Verbatim}[commandchars=\\\{\}]

\end{Verbatim}
\end{tcolorbox}

    Depois calcule as duas \textbf{inversas} (na lista foi pedido apenas de
A) e \textbf{verifique se elas comutam}.

    \begin{tcolorbox}[breakable, size=fbox, boxrule=1pt, pad at break*=1mm,colback=cellbackground, colframe=cellborder]
\prompt{In}{incolor}{ }{\boxspacing}
\begin{Verbatim}[commandchars=\\\{\}]

\end{Verbatim}
\end{tcolorbox}

    Percebe agora os " \textbf{super poderes} " que isso te dá??

Poderia ser uma matriz de 10x10 ou 1000x1000 que o \(\underline{seu}\)
trabalho seria essencialmente o mesmo! 😉

\emph{Numpy} faria todo o trabalho pesado para você!! Parece-me
motivação suficiente para aprender usar, não?

    \hypertarget{exemplo-3}{%
\subsection{Exemplo 3}\label{exemplo-3}}

Como último exemplo, vamos considerar agora o uma aplicação de
distribuição estatística. Por exemplo, considere que foram feitas várias
medidas de um observável o resultado é dado por uma lista de valores,
como a abaixo.

    \begin{tcolorbox}[breakable, size=fbox, boxrule=1pt, pad at break*=1mm,colback=cellbackground, colframe=cellborder]
\prompt{In}{incolor}{50}{\boxspacing}
\begin{Verbatim}[commandchars=\\\{\}]
\PY{n}{m} \PY{o}{=} \PY{p}{[}\PY{l+m+mi}{10}\PY{p}{,}\PY{l+m+mi}{13}\PY{p}{,}\PY{l+m+mi}{14}\PY{p}{,}\PY{l+m+mi}{14}\PY{p}{,}\PY{l+m+mi}{6}\PY{p}{,}\PY{l+m+mi}{8}\PY{p}{,}\PY{l+m+mi}{7}\PY{p}{,}\PY{l+m+mi}{9}\PY{p}{,}\PY{l+m+mi}{12}\PY{p}{,}\PY{l+m+mi}{14}\PY{p}{,}\PY{l+m+mi}{13}\PY{p}{,}\PY{l+m+mi}{11}\PY{p}{,}\PY{l+m+mi}{10}\PY{p}{,}\PY{l+m+mi}{7}\PY{p}{,}\PY{l+m+mi}{7}\PY{p}{]}
\PY{n}{vetor} \PY{o}{=} \PY{n}{array}\PY{p}{(}\PY{n}{m}\PY{p}{)}\PY{o}{/}\PY{l+m+mi}{2}    \PY{c+c1}{\PYZsh{}importante converter para array primeiro!}
\PY{n+nb}{print}\PY{p}{(}\PY{n}{m}\PY{p}{)}
\PY{n+nb}{print}\PY{p}{(}\PY{n}{vetor}\PY{p}{)}   \PY{c+c1}{\PYZsh{} usando a função print() para mostrar os valores das variáveis m e vetor}
\end{Verbatim}
\end{tcolorbox}

    \begin{Verbatim}[commandchars=\\\{\}]
[10, 13, 14, 14, 6, 8, 7, 9, 12, 14, 13, 11, 10, 7, 7]
[5.  6.5 7.  7.  3.  4.  3.5 4.5 6.  7.  6.5 5.5 5.  3.5 3.5]
    \end{Verbatim}

    \begin{tcolorbox}[breakable, size=fbox, boxrule=1pt, pad at break*=1mm,colback=cellbackground, colframe=cellborder]
\prompt{In}{incolor}{51}{\boxspacing}
\begin{Verbatim}[commandchars=\\\{\}]
\PY{c+c1}{\PYZsh{} podemos construi um histograma das medidas usando a função hist()}
\PY{n}{n1}\PY{p}{,} \PY{n}{bins1}\PY{p}{,} \PY{n}{patches1} \PY{o}{=} \PY{n}{hist}\PY{p}{(}\PY{n}{m}\PY{p}{,}\PY{n}{bins}\PY{o}{=}\PY{l+m+mi}{5}\PY{p}{,}\PY{n+nb}{range}\PY{o}{=}\PY{p}{(}\PY{l+m+mi}{5}\PY{p}{,}\PY{l+m+mi}{14}\PY{p}{)}\PY{p}{)}
\end{Verbatim}
\end{tcolorbox}

    \begin{center}
    \adjustimage{max size={0.9\linewidth}{0.9\paperheight}}{output_89_0.png}
    \end{center}
    { \hspace*{\fill} \\}
    
    \begin{tcolorbox}[breakable, size=fbox, boxrule=1pt, pad at break*=1mm,colback=cellbackground, colframe=cellborder]
\prompt{In}{incolor}{52}{\boxspacing}
\begin{Verbatim}[commandchars=\\\{\}]
\PY{c+c1}{\PYZsh{} a frequência (\PYZsh{} de ocorrências) em cada bin é dada por}
\PY{n}{n1}
\end{Verbatim}
\end{tcolorbox}

            \begin{tcolorbox}[breakable, size=fbox, boxrule=.5pt, pad at break*=1mm, opacityfill=0]
\prompt{Out}{outcolor}{52}{\boxspacing}
\begin{Verbatim}[commandchars=\\\{\}]
array([1., 4., 3., 2., 5.])
\end{Verbatim}
\end{tcolorbox}
        
    \begin{tcolorbox}[breakable, size=fbox, boxrule=1pt, pad at break*=1mm,colback=cellbackground, colframe=cellborder]
\prompt{In}{incolor}{53}{\boxspacing}
\begin{Verbatim}[commandchars=\\\{\}]
\PY{c+c1}{\PYZsh{} observe que m é uma lista enquanto vetor é um array, mas isso não faz diferença p/ hist()}
\PY{n}{n2}\PY{p}{,} \PY{n}{bins2}\PY{p}{,} \PY{n}{patches2} \PY{o}{=} \PY{n}{hist}\PY{p}{(}\PY{n}{vetor}\PY{p}{,}\PY{n}{bins}\PY{o}{=}\PY{l+m+mi}{5}\PY{p}{,}\PY{n+nb}{range}\PY{o}{=}\PY{p}{(}\PY{l+m+mi}{1}\PY{p}{,}\PY{l+m+mi}{14}\PY{p}{)}\PY{p}{)}
\end{Verbatim}
\end{tcolorbox}

    \begin{center}
    \adjustimage{max size={0.9\linewidth}{0.9\paperheight}}{output_91_0.png}
    \end{center}
    { \hspace*{\fill} \\}
    
    \begin{tcolorbox}[breakable, size=fbox, boxrule=1pt, pad at break*=1mm,colback=cellbackground, colframe=cellborder]
\prompt{In}{incolor}{54}{\boxspacing}
\begin{Verbatim}[commandchars=\\\{\}]
\PY{n}{n2}
\end{Verbatim}
\end{tcolorbox}

            \begin{tcolorbox}[breakable, size=fbox, boxrule=.5pt, pad at break*=1mm, opacityfill=0]
\prompt{Out}{outcolor}{54}{\boxspacing}
\begin{Verbatim}[commandchars=\\\{\}]
array([4., 6., 5., 0., 0.])
\end{Verbatim}
\end{tcolorbox}
        
    Podemos agora facilmente calcular a probabilidade (a partir das medidas,
não a teórica) dos resultados dessas medidas.

    \begin{tcolorbox}[breakable, size=fbox, boxrule=1pt, pad at break*=1mm,colback=cellbackground, colframe=cellborder]
\prompt{In}{incolor}{55}{\boxspacing}
\begin{Verbatim}[commandchars=\\\{\}]
\PY{n}{pvals1} \PY{o}{=} \PY{n}{n1}\PY{o}{/}\PY{n}{n1}\PY{o}{.}\PY{n}{sum}\PY{p}{(}\PY{p}{)}
\PY{n}{pvals1}
\end{Verbatim}
\end{tcolorbox}

            \begin{tcolorbox}[breakable, size=fbox, boxrule=.5pt, pad at break*=1mm, opacityfill=0]
\prompt{Out}{outcolor}{55}{\boxspacing}
\begin{Verbatim}[commandchars=\\\{\}]
array([0.06666667, 0.26666667, 0.2       , 0.13333333, 0.33333333])
\end{Verbatim}
\end{tcolorbox}
        
    \begin{tcolorbox}[breakable, size=fbox, boxrule=1pt, pad at break*=1mm,colback=cellbackground, colframe=cellborder]
\prompt{In}{incolor}{56}{\boxspacing}
\begin{Verbatim}[commandchars=\\\{\}]
\PY{n}{pvals2} \PY{o}{=} \PY{n}{n2}\PY{o}{/}\PY{n}{n2}\PY{o}{.}\PY{n}{sum}\PY{p}{(}\PY{p}{)}
\PY{n}{pvals2}
\end{Verbatim}
\end{tcolorbox}

            \begin{tcolorbox}[breakable, size=fbox, boxrule=.5pt, pad at break*=1mm, opacityfill=0]
\prompt{Out}{outcolor}{56}{\boxspacing}
\begin{Verbatim}[commandchars=\\\{\}]
array([0.26666667, 0.4       , 0.33333333, 0.        , 0.        ])
\end{Verbatim}
\end{tcolorbox}
        
    \begin{tcolorbox}[breakable, size=fbox, boxrule=1pt, pad at break*=1mm,colback=cellbackground, colframe=cellborder]
\prompt{In}{incolor}{57}{\boxspacing}
\begin{Verbatim}[commandchars=\\\{\}]
\PY{c+c1}{\PYZsh{} note que as somas estão normalizadas...}
\PY{n+nb}{print}\PY{p}{(}\PY{l+s+s2}{\PYZdq{}}\PY{l+s+s2}{A soma de todas as probabilidades pvals1 =}\PY{l+s+s2}{\PYZdq{}}\PY{p}{,}\PY{n}{np}\PY{o}{.}\PY{n}{sum}\PY{p}{(}\PY{n}{pvals1}\PY{p}{)}\PY{p}{)}
\PY{n+nb}{print}\PY{p}{(}\PY{l+s+s2}{\PYZdq{}}\PY{l+s+s2}{A soma de todas as probabilidades pvals2 =}\PY{l+s+s2}{\PYZdq{}}\PY{p}{,}\PY{n}{np}\PY{o}{.}\PY{n}{sum}\PY{p}{(}\PY{n}{pvals2}\PY{p}{)}\PY{p}{)}
\end{Verbatim}
\end{tcolorbox}

    \begin{Verbatim}[commandchars=\\\{\}]
A soma de todas as probabilidades pvals1 = 1.0
A soma de todas as probabilidades pvals2 = 1.0
    \end{Verbatim}

    Vamos nos aprofundar um pouco mais nisso, pois vale a pena\ldots{}

    \begin{tcolorbox}[breakable, size=fbox, boxrule=1pt, pad at break*=1mm,colback=cellbackground, colframe=cellborder]
\prompt{In}{incolor}{58}{\boxspacing}
\begin{Verbatim}[commandchars=\\\{\}]
\PY{c+c1}{\PYZsh{} o valore médio de \PYZlt{}m\PYZgt{}}
\PY{n}{m} \PY{o}{=}  \PY{l+m+mi}{2}\PY{o}{*}\PY{n}{vetor} 
\PY{n}{sm} \PY{o}{=} \PY{n}{m}\PY{o}{.}\PY{n}{sum}\PY{p}{(}\PY{p}{)}
\PY{n}{sm}
\end{Verbatim}
\end{tcolorbox}

            \begin{tcolorbox}[breakable, size=fbox, boxrule=.5pt, pad at break*=1mm, opacityfill=0]
\prompt{Out}{outcolor}{58}{\boxspacing}
\begin{Verbatim}[commandchars=\\\{\}]
155.0
\end{Verbatim}
\end{tcolorbox}
        
    \begin{tcolorbox}[breakable, size=fbox, boxrule=1pt, pad at break*=1mm,colback=cellbackground, colframe=cellborder]
\prompt{In}{incolor}{59}{\boxspacing}
\begin{Verbatim}[commandchars=\\\{\}]
\PY{n}{np}\PY{o}{.}\PY{n}{sum}\PY{p}{(}\PY{n}{m}\PY{p}{)}
\end{Verbatim}
\end{tcolorbox}

            \begin{tcolorbox}[breakable, size=fbox, boxrule=.5pt, pad at break*=1mm, opacityfill=0]
\prompt{Out}{outcolor}{59}{\boxspacing}
\begin{Verbatim}[commandchars=\\\{\}]
155.0
\end{Verbatim}
\end{tcolorbox}
        
    \begin{tcolorbox}[breakable, size=fbox, boxrule=1pt, pad at break*=1mm,colback=cellbackground, colframe=cellborder]
\prompt{In}{incolor}{60}{\boxspacing}
\begin{Verbatim}[commandchars=\\\{\}]
\PY{n}{p} \PY{o}{=} \PY{l+m+mi}{2}\PY{o}{*}\PY{n}{vetor}\PY{o}{/}\PY{n}{sm}
\PY{n}{m1} \PY{o}{=} \PY{n}{dot}\PY{p}{(}\PY{n}{p}\PY{p}{,}\PY{n}{m}\PY{p}{)}  \PY{c+c1}{\PYZsh{} \PYZlt{}m\PYZgt{} : valor médio de m}
\PY{n}{m1}
\end{Verbatim}
\end{tcolorbox}

            \begin{tcolorbox}[breakable, size=fbox, boxrule=.5pt, pad at break*=1mm, opacityfill=0]
\prompt{Out}{outcolor}{60}{\boxspacing}
\begin{Verbatim}[commandchars=\\\{\}]
11.09032258064516
\end{Verbatim}
\end{tcolorbox}
        
    \[
\langle m \rangle = \sum_i p_i\ m_i
\]

\[
\langle m \rangle^2 = \sum_i p_i\ m_i^2
\]

\[
\sigma^2 = \langle m^2 \rangle - \langle m \rangle^2
\]

    \begin{tcolorbox}[breakable, size=fbox, boxrule=1pt, pad at break*=1mm,colback=cellbackground, colframe=cellborder]
\prompt{In}{incolor}{61}{\boxspacing}
\begin{Verbatim}[commandchars=\\\{\}]
\PY{n}{m2} \PY{o}{=} \PY{n}{dot}\PY{p}{(}\PY{n}{p}\PY{p}{,}\PY{n}{m}\PY{o}{*}\PY{o}{*}\PY{l+m+mi}{2}\PY{p}{)}
\PY{n}{m2}
\end{Verbatim}
\end{tcolorbox}

            \begin{tcolorbox}[breakable, size=fbox, boxrule=.5pt, pad at break*=1mm, opacityfill=0]
\prompt{Out}{outcolor}{61}{\boxspacing}
\begin{Verbatim}[commandchars=\\\{\}]
130.13548387096773
\end{Verbatim}
\end{tcolorbox}
        
    \begin{tcolorbox}[breakable, size=fbox, boxrule=1pt, pad at break*=1mm,colback=cellbackground, colframe=cellborder]
\prompt{In}{incolor}{62}{\boxspacing}
\begin{Verbatim}[commandchars=\\\{\}]
\PY{n}{sigma} \PY{o}{=} \PY{n}{sqrt}\PY{p}{(} \PY{n}{m2} \PY{o}{\PYZhy{}} \PY{n}{m1}\PY{o}{*}\PY{o}{*}\PY{l+m+mi}{2} \PY{p}{)}
\PY{n}{sigma}
\end{Verbatim}
\end{tcolorbox}

            \begin{tcolorbox}[breakable, size=fbox, boxrule=.5pt, pad at break*=1mm, opacityfill=0]
\prompt{Out}{outcolor}{62}{\boxspacing}
\begin{Verbatim}[commandchars=\\\{\}]
2.6721206799468824
\end{Verbatim}
\end{tcolorbox}
        
    \begin{tcolorbox}[breakable, size=fbox, boxrule=1pt, pad at break*=1mm,colback=cellbackground, colframe=cellborder]
\prompt{In}{incolor}{63}{\boxspacing}
\begin{Verbatim}[commandchars=\\\{\}]
\PY{n}{pm} \PY{o}{=} \PY{n}{np}\PY{o}{.}\PY{n}{random}\PY{o}{.}\PY{n}{normal}\PY{p}{(}\PY{n}{m1}\PY{p}{,} \PY{n}{sigma}\PY{p}{,} \PY{l+m+mi}{10000}\PY{p}{)}
\PY{n}{hist}\PY{p}{(}\PY{n}{pm}\PY{p}{,}\PY{l+m+mi}{100}\PY{p}{)}\PY{p}{;}
\end{Verbatim}
\end{tcolorbox}

    \begin{center}
    \adjustimage{max size={0.9\linewidth}{0.9\paperheight}}{output_104_0.png}
    \end{center}
    { \hspace*{\fill} \\}
    
    \[
p(x)=\frac{1}{\sigma \sqrt{2\pi} } \exp \left(\frac{-(x - \langle m \rangle)^2}{2 \sigma^2} \right)
\]

    \begin{tcolorbox}[breakable, size=fbox, boxrule=1pt, pad at break*=1mm,colback=cellbackground, colframe=cellborder]
\prompt{In}{incolor}{64}{\boxspacing}
\begin{Verbatim}[commandchars=\\\{\}]
\PY{k+kn}{import} \PY{n+nn}{matplotlib}\PY{n+nn}{.}\PY{n+nn}{pyplot} \PY{k}{as} \PY{n+nn}{plt}
\PY{n}{count}\PY{p}{,} \PY{n}{bins}\PY{p}{,} \PY{n}{ignored} \PY{o}{=} \PY{n}{plt}\PY{o}{.}\PY{n}{hist}\PY{p}{(}\PY{n}{pm}\PY{p}{,} \PY{l+m+mi}{100}\PY{p}{,} \PY{n}{density}\PY{o}{=}\PY{k+kc}{True}\PY{p}{)}
\PY{n}{plt}\PY{o}{.}\PY{n}{plot}\PY{p}{(}\PY{n}{bins}\PY{p}{,} \PY{l+m+mi}{1}\PY{o}{/}\PY{p}{(}\PY{n}{sigma} \PY{o}{*} \PY{n}{np}\PY{o}{.}\PY{n}{sqrt}\PY{p}{(}\PY{l+m+mi}{2} \PY{o}{*} \PY{n}{np}\PY{o}{.}\PY{n}{pi}\PY{p}{)}\PY{p}{)} \PY{o}{*}
               \PY{n}{np}\PY{o}{.}\PY{n}{exp}\PY{p}{(} \PY{o}{\PYZhy{}} \PY{p}{(}\PY{n}{bins} \PY{o}{\PYZhy{}} \PY{n}{m1}\PY{p}{)}\PY{o}{*}\PY{o}{*}\PY{l+m+mi}{2} \PY{o}{/} \PY{p}{(}\PY{l+m+mi}{2} \PY{o}{*} \PY{n}{sigma}\PY{o}{*}\PY{o}{*}\PY{l+m+mi}{2}\PY{p}{)} \PY{p}{)}\PY{p}{,}
         \PY{n}{linewidth}\PY{o}{=}\PY{l+m+mi}{2}\PY{p}{,} \PY{n}{color}\PY{o}{=}\PY{l+s+s1}{\PYZsq{}}\PY{l+s+s1}{r}\PY{l+s+s1}{\PYZsq{}}\PY{p}{)}\PY{p}{;}
\end{Verbatim}
\end{tcolorbox}

    \begin{center}
    \adjustimage{max size={0.9\linewidth}{0.9\paperheight}}{output_106_0.png}
    \end{center}
    { \hspace*{\fill} \\}
    
    \textbf{Sugestão:} tenta agora usar esses dados para calcular a
probabilidade de se obter um determinado valor \(m_k\) (ou, mais
estritamente falando, um intervalo entre \(m_k\) e \(m_k + dm\).). Agora
você teria, provavelmente, muito mais chances de fazer uma boa
comparação com os valores teóricos calculados (analiticamente) para um
operador observável físico.

    \begin{tcolorbox}[breakable, size=fbox, boxrule=1pt, pad at break*=1mm,colback=cellbackground, colframe=cellborder]
\prompt{In}{incolor}{ }{\boxspacing}
\begin{Verbatim}[commandchars=\\\{\}]

\end{Verbatim}
\end{tcolorbox}

    \begin{tcolorbox}[breakable, size=fbox, boxrule=1pt, pad at break*=1mm,colback=cellbackground, colframe=cellborder]
\prompt{In}{incolor}{ }{\boxspacing}
\begin{Verbatim}[commandchars=\\\{\}]

\end{Verbatim}
\end{tcolorbox}


    % Add a bibliography block to the postdoc
    
    
    
\end{document}
